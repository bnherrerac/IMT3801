\documentclass{article}
\usepackage[utf8]{inputenc}
\usepackage{amsmath, amsthm, amssymb, mathpazo, isomath, mathtools}
\usepackage{subcaption,graphicx,pgfplots}
\usepackage{fullpage}
\usepackage{booktabs}
\usepackage{hyperref}
\usepackage{algorithm, algorithmic}
\usepackage{mathtools}
\usepackage{todonotes}

\title{Tarea 2}
%\author{Nicol\'as A Barnafi\thanks{Instituto de Ingeniería Biológica y Médica, Pontificia Universidad Católica de Chile, Chile}, Axel Osses\thanks{Departamento de Ingeniería Matemática, Universidad de Chile, Chile}}
%\author{Nicol\'as A Barnafi}
\date{}

\renewcommand{\vec}{\vectorsym}
\newcommand{\mat}{\matrixsym}
\newcommand{\ten}{\tensorsym}
\DeclareMathOperator{\grad}{\nabla}
\DeclareMathOperator{\dive}{\text{div}}
\DeclareMathOperator{\curl}{\text{curl}}
\DeclareMathOperator{\tr}{\text{tr}}
\DeclareMathOperator{\sym}{\text{sym}}
\newtheorem{remark}{Remark}
\newtheorem{definition}{Definition}
\newcommand{\R}{\mathbb{R}}
\newcommand{\D}{\mathcal{D}}

\newcommand{\tin}{\text{in}}
\newcommand{\ton}{\text{on}}

\newtheorem{theorem}{Theorem}
\newtheorem{lemma}{Lemma}
\newcommand{\pts}[1]{[{\bf #1 puntos}] }

\begin{document}

\maketitle
\hfill \textbf{Fecha de entrega: 23:59 del 25/04/2025}
 
\todo[inline,color=white!90!black]{\textbf{Instrucciones: } La tarea debe ser entregada de manera individual en un informe en formato .pdf a través del buzón habilitado en la plataforma Canvas, donde deben mostrar también el código desarrollado. Para su conveniencia, pueden entregar las tareas en un Jupyter Notebook, de modo que sea más cómodo mostrar el código. La política de atrasos será: se calculará un factor lineal que vale 1 a la hora de entrega y 0 48 horas después. Esto multiplicará su puntaje obtenido. Pueden usar ChatGPT u otros modelos solo a conciencia. El uso de salidas de GPT sin su debida comprensión será severamente sancionado. }

\begin{enumerate}
    \item Considere el problema biestable con difusión en el tiempo y espacio $\Omega_T = (0,T)\times (0,1)$:
            $$ \dot u - \mu \partial_{xx} u - u(1-u^2) = f. $$
            Como datos y condiciones de borde, considere:
            \begin{itemize}
                \item $f(t,x) = I_{x\in (0.4,0.6)}I_{t<1}$, donde $I$ es una función indicatriz.
                \item Una condición inicial dada por $u(0,x) = 0$ para todo $x$ in $(0,1)$. 
                \item Condiciones de Neumann homogéneas en el borde para todo $t>0$. 
            \end{itemize}
            Para este modelo:
            \begin{enumerate}
                \item\pts{1} Proponga un esquema de discretización y demuestre que es consistente.
                \item\pts{1} Usando el análisis de von Neumann, proponga hipótesis de la solución discreta y los parámetros de discretización que permitan obtener una aproximación estable, y por lo tanto convergente.
                \item\pts{1} Proponga tres formulaciones de discretización en tiempo: (i) explícita, (ii) implícita, y (iii) semi-implícita. 
                \item\pts{1} Formule un método iterativo (Newton o punto fijo) para resolver la formulación implícita y comente sobre las propiedades de convergencia que tiene el método propuesto. 
                \item\pts{3} Implemente el método implícito propuesto y grafique la evolución de la solución para $\mu=0.1$. Comente sus resultados.
                \item\pts{2} Implemente el método semi-implícito propuesto y grafique la evolución de la solución para $\mu=0.1$. Comente sus resultados y compárelos con los obtenidos en el punto anterior.
            \end{enumerate}

    \item En esta pregunta, jugaremos con la definición de las distribuciones. Usaremos la notación $\mathcal D(\R) = C_0^\infty(\R)$, espacio de funciones suaves con soporte compacto en $\R$.
        \begin{enumerate}
            \item\pts{1} Definimos la inyección de funciones integrables a distribuciones $T:L^1(\R) \to \left(\mathcal D(\R)\right)'$ según la acción
                $$ (Tf)(\varphi)  = \left\langle Tf, \varphi\right\rangle_{\mathcal D'\times \mathcal D}.$$
                Demuestre que dicha inyección es continua. 
            \item\pts{1}Caracterice la convergencia en el espacio de distribuciones $\left(\mathcal D(\R)\right)'$. 
            \item\pts{1}Demuestre que, dado $x_0\in \R$,  la sucesión inducida por la función $I_n(x) = 2nI_{(x_0-1/2n, x_0+1/2n)}$, i.e. $\{I_n\}_n$, converge como distribución a la delta de Dirac $\delta_{x_0}$. 
            \item\pts{1}Demuestre que no existe ninguna función $f$ en $L^1$ tal que $T_f = \delta_0$. 
            \item\pts{2} Sabemos que dada una función vectorial $F:\Omega \to \R^d$, su divergencia está dada por $\dive F=\sum_i \partial_{x_i}F$. Para extender esta noción a distribuciones, demuestre dado un dominio $\Omega$ los espacios $(\mathcal D(\Omega, \R^d))'$ y $[(\mathcal D(\Omega))']^d$ son homeomorfos\footnote{$X^3 = X\times X\times X$} y con ello construya una definición de divergencia en $(\mathcal D(\Omega, \R^d))'$. Hint: $\int_\Omega (\dive F)\varphi = \int_{\partial\Omega} \varphi F\cdot \vec n - \int_\Omega F\cdot \grad \varphi$. 
            \item\pts{2} Extienda la construcción de la pregunta anterior para definir un $\curl$ distribucional.
            \item\pts{2} Sea $\vec e_j$ el $j$-ésimo vector canónico en $\R^d$. Definimos el operador de diferencias finitas parciales de paso $h$ como
                $$ D_j^hf(\vec x) \coloneqq \frac{f(x+h\vec e_j) - f(x)}{h}. $$
                Demuestre que para cada $f$ en $\mathcal D(\R^d)$, se tiene que $D_j^hf $ converge a $\frac{\partial f}{\partial x_j}$ en la topología de $\mathcal D(\R^d)$ cuando $h$ va a 0.  
            \item\pts{2} Considere la función $\Phi:\R^3\to \mathbb C$ dada por
                    $$ \Phi(x) = \frac{1}{4\pi|x|} e^{-ik |x|}. $$
                Muestre que se tiene la siguiente igualdad en $\mathcal D'(\R^3)$:
                        $$ \Delta \Phi + k^2 \Phi = \delta_0, $$
                i.e. en el sentido de las distribuciones, donde $\delta_0$ es la delta de Dirac en $x=0$. 
        \end{enumerate}
\end{enumerate}

\todo[inline,color=white!90!black]{\textbf{Nota: } Abriremos un foro en Canvas para revisar cualquier typo y/o error que haya en el enunciado.}
\end{document}

