\documentclass{article}
\usepackage[utf8]{inputenc}
\usepackage{amsmath, amsthm, amssymb, mathpazo, isomath, mathtools}
\usepackage{subcaption,graphicx,pgfplots}
\usepackage{fullpage}
\usepackage{booktabs}
\usepackage{hyperref}
\usepackage{algorithm, algorithmic}
\usepackage{mathtools}
\usepackage{todonotes}

\title{Examen}
%\author{Nicol\'as A Barnafi\thanks{Instituto de Ingeniería Biológica y Médica, Pontificia Universidad Católica de Chile, Chile}, Axel Osses\thanks{Departamento de Ingeniería Matemática, Universidad de Chile, Chile}}
%\author{Nicol\'as A Barnafi}
\date{}

\renewcommand{\vec}{\vectorsym}
\newcommand{\mat}{\matrixsym}
\newcommand{\ten}{\tensorsym}
\DeclareMathOperator{\grad}{\nabla}
\DeclareMathOperator{\dive}{\text{div}}
\DeclareMathOperator{\curl}{\text{curl}}
\DeclareMathOperator{\tr}{\text{tr}}
\DeclareMathOperator{\sym}{\text{sym}}
\newtheorem{remark}{Remark}
\newtheorem{definition}{Definition}
\newcommand{\R}{\mathbb{R}}
\newcommand{\D}{\mathcal{D}}
\newcommand{\parder}[2]{\frac{\partial\,#1}{\partial\,#2}}

\newcommand{\tin}{\text{in}}
\newcommand{\ton}{\text{on}}

\newtheorem{theorem}{Theorem}
\newtheorem{lemma}{Lemma}
\newcommand{\pts}[1]{[{\bf #1 puntos}] }

\begin{document}

\maketitle
\hfill \textbf{Fecha de entrega: 23:59 del 30/06/2025}
 
\todo[inline,color=white!90!black]{\textbf{Instrucciones: } La tarea debe ser entregada de manera individual en un informe en formato .pdf a través del buzón habilitado en la plataforma Canvas, donde deben mostrar también el código desarrollado. Para su conveniencia, pueden entregar las tareas en un Jupyter Notebook, de modo que sea más cómodo mostrar el código. En cualquier caso, se debe entregar un único archivo como respuesta a la tarea. Entregas atrasadas tienen un 1.0 automáticamente. Pueden usar ChatGPT u otros modelos solo a conciencia. El uso de salidas de GPT sin su debida comprensión será severamente sancionado. Todas las preguntas tienen un punto, y el puntaje estara dado por: 1.0 punto si la pregunta esta correcta, 0.5 si tiene fallas menores pero esta principalmente bien, 0.0 si presenta errores conceptuales importantes.}

\begin{enumerate}

    \item Considere el problema ADR con condiciones de borde homogéneas:
            $$ \begin{aligned}
                -\mu \Delta u + \vec b \cdot \grad u + c u &= f \qquad \Omega \\
                u &= 0 \qquad \partial\Omega.
            \end{aligned}$$
            Asumiendo que $\mu,\vec b,c$ son funciones, considere la discretización de este problema en el dominio $\Omega = (0,1)$. 
            \begin{itemize}
                \item\pts{1} Muestre que, bajo hipótesis adecuadas sobre los parámetros,  este problema tiene una única solución débil y que además esta es continua.
                \item\pts{1}Describa cómo aproximar este problema con diferencias finitas y escriba explícitamente la matriz discreta.
                \item\pts{1}Describa cómo aproximar este problema con elementos finitos (de primer orden) y escriba explícitamente la matriz discreta. 
                \item\pts{1}Existe algún rango de parámetros donde las discretizaciones resulten en el mismo problema? Y parámetros donde sean diferentes? 
                \item\pts{1}Explique las garantías teóricas que tienen diferencias finitas y elementos finitos, y compárelas. Qué le parece más conveniente en este caso? 
            \end{itemize}

    \item Considere el problema biarmónico dado por el bilaplaciano:
            $$ \Delta^2 u = f \qquad \Omega, $$
            con condición de borde $u = \grad u\cdot \vec n = 0$ en $\partial\Omega$, y considere el espacio funcional 
                $$ H_0^2(\Omega) = \{v \in H^2(\Omega): v=\grad v \cdot \vec n = 0 \quad \partial\Omega\}. $$
            \begin{itemize}
                \item\pts{1} Demuestre que $v\mapsto|\Delta v|$ es una norma .
                \item\pts{1} Demuestre que $v\mapsto|\Delta v|$ es una norma equivalente a la norma natural de $H^2(\Omega)$ en $H_0^2(\Omega)$.
                \item\pts{1} Encuentre la formulación variacional del problema y diga cuál es la regularidad mínima que requiere para $f$. 
                \item\pts{1} Demuestre que el problema tiene existencia y unicidad de soluciones. 
            \end{itemize}
        
    \item Considere la ecuación de conservación de masa con una fuente de masa $\theta$: 
            $$ \parder{\rho}{t} + \dive(\rho \vec v) = \theta. $$
            \begin{itemize}
                \item\pts{1} Muestre cómo modificaría la deducción de la ley de conservación de masa para considerar el término fuente de masa $\theta$ y obtener la ecuacion descrita. 
                \item\pts{1} Suponga un material donde la velocidad de masa $\rho\vec v$ está dada por el inverso del gradiente de densidad: 
                        $$ \rho\vec v = - \ten K \grad \rho, $$
                        donde $\ten K$ es un tensor de segundo orden. Muestre que la ecuación resultante es la ecuación del calor para $\rho$ y que la formulación débil para condiciones de Dirichlet homogéneas es: Hallar $\rho(t)$\footnote{Los espacios de funciones tales que están en un espacio de Sobolev en cada instante se llaman espacios de Bochner. Puede ignorar esta dificultad técnica y asumir que es sensato escribir una función $\rho$ tal que para cada instante $t$ se tiene que $\rho(t)$ pertenece a un espacio de Sobolev.} en $H_0^1(\Omega)$ dada una condición inicial $\rho_0$ tal que 
                        $$ \left(\partial_t \rho, v\right) + (\ten K \grad \rho, \grad v) = (\theta, v) \qquad \forall v \in H_0^1(\Omega). $$
                \item\pts{1} Considere una discretización por diferencias finitas implicitas en el tiempo, y muestre cuál es el operador diferencial que aparece en el problema de cada instante $t^n$. Muestre que la formulación débil que obtiene para cada instante $t^n$ es encontrar $\rho^n$ en $H_0^1(\Omega)$ (asumiendo condiciones de Dirichlet homogéneas) tal que 
                    $$ \left(\frac{\rho^n - \rho^{n-1}}{\Delta t}, v\right) + (\ten K \grad \rho^n, \grad v) = (\theta(t^n), v) \qquad \forall v\in H_0^1(\Omega). $$ 
                    Defina claramente todos los objetos matemáticos usados para que esta aproximación esté rigurosamente justificada. Replique el cálculo para una discretización en tiempo dada por el método $\theta$. A problemas como este donde solo una de las variables está discretizada se les conoce como problemas \emph{semi-discretos}. 
                \item\pts{1} Estudie la invertibilidad del sistema discreto en cada instante $t^n$ para $\theta=0$, $\theta=1/2$, y $\theta=1$. Use el Lema de Lax-Milgram para justificar su respuesta. 
                \item\pts{1} La condición inf-sup ayuda a caracterizar la sobreyectividad de un operador. Tomando como motivación el ejercicio anterior, explique si es posible establecer una condición inf-sup para el siguiente problema: Encontrar $u$ en $H_0^1(\Omega)$ tal que 
                        $$ (u, v)_0 = \langle f, v\rangle_{H^{-1}\times H_0^1} \qquad \forall v\in H_0^1(\Omega). $$
                        Naturalmente, no es posible usar Lax-Milgram ya que el producto en $L^2$ no es elíptico en $H_0^1$. 
                \item\pts{1} Proponga un esquema de elementos finitos que le permita aproximar el problema semi-discreto a partir de uno completamente discreto. 
            \end{itemize}
\end{enumerate}

\todo[inline,color=white!90!black]{\textbf{Nota: } Abriremos un foro en Canvas para revisar cualquier typo y/o error que haya en el enunciado.}
\end{document}

