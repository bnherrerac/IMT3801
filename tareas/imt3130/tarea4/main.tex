\documentclass{article}
\usepackage[utf8]{inputenc}
\usepackage{amsmath, amsthm, amssymb, mathpazo, isomath, mathtools}
\usepackage{subcaption,graphicx,pgfplots}
\usepackage{fullpage}
\usepackage{booktabs}
\usepackage{hyperref}
\usepackage{algorithm, algorithmic}
\usepackage{mathtools}
\usepackage{todonotes}

\title{Tarea 4}
%\author{Nicol\'as A Barnafi\thanks{Instituto de Ingeniería Biológica y Médica, Pontificia Universidad Católica de Chile, Chile}, Axel Osses\thanks{Departamento de Ingeniería Matemática, Universidad de Chile, Chile}}
%\author{Nicol\'as A Barnafi}
\date{}

\renewcommand{\vec}{\vectorsym}
\newcommand{\mat}{\matrixsym}
\newcommand{\ten}{\tensorsym}
\DeclareMathOperator{\grad}{\nabla}
\DeclareMathOperator{\dive}{\text{div}}
\DeclareMathOperator{\curl}{\text{curl}}
\DeclareMathOperator{\tr}{\text{tr}}
\DeclareMathOperator{\sym}{\text{sym}}
\newtheorem{remark}{Remark}
\newtheorem{definition}{Definition}
\newcommand{\R}{\mathbb{R}}
\newcommand{\D}{\mathcal{D}}

\newcommand{\tin}{\text{in}}
\newcommand{\ton}{\text{on}}

\newtheorem{theorem}{Theorem}
\newtheorem{lemma}{Lemma}
\newcommand{\pts}[1]{[{\bf #1 puntos}] }

\begin{document}

\maketitle
\hfill \textbf{Fecha de entrega: 23:59 del 30/05/2025}
 
\todo[inline,color=white!90!black]{\textbf{Instrucciones: } La tarea debe ser entregada de manera individual en un informe en formato .pdf a través del buzón habilitado en la plataforma Canvas, donde deben mostrar también el código desarrollado. Para su conveniencia, pueden entregar las tareas en un Jupyter Notebook, de modo que sea más cómodo mostrar el código. En cualquier caso, se debe entregar un único archivo como respuesta a la tarea. La política de atrasos será: se calculará un factor lineal que vale 1 a la hora de entrega y 0 48 horas después. Esto multiplicará su puntaje obtenido. Pueden usar ChatGPT u otros modelos solo a conciencia. El uso de salidas de GPT sin su debida comprensión será severamente sancionado. }

\begin{enumerate}

    \item Dado un espacio de Hilbert $H$, considere una forma bilineal $a:H\times H\to \R$ acotada y elíptica. Demuestre que también satisface las hipótesis del Lema Generalizado de Lax-Milgra (Teorema de Ne\v cas-Babu\v ska-Brezzi). 

    \item Considere el problema de Stokes, dado por:

    \item Muestre que LBB implica NBB. 

    \item Mostrar convergencia de ADR.

    \item Implementar ADR y verificar tasas de convergencia.

\end{enumerate}

\todo[inline,color=white!90!black]{\textbf{Nota: } Abriremos un foro en Canvas para revisar cualquier typo y/o error que haya en el enunciado.}
\end{document}

