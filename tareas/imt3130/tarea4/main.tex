\documentclass{article}
\usepackage[utf8]{inputenc}
\usepackage{amsmath, amsthm, amssymb, mathpazo, isomath, mathtools}
\usepackage{subcaption,graphicx,pgfplots}
\usepackage{fullpage}
\usepackage{booktabs}
\usepackage{hyperref}
\usepackage{algorithm, algorithmic}
\usepackage{mathtools}
\usepackage{todonotes}

\title{Tarea 4}
%\author{Nicol\'as A Barnafi\thanks{Instituto de Ingeniería Biológica y Médica, Pontificia Universidad Católica de Chile, Chile}, Axel Osses\thanks{Departamento de Ingeniería Matemática, Universidad de Chile, Chile}}
%\author{Nicol\'as A Barnafi}
\date{}

\renewcommand{\vec}{\vectorsym}
\newcommand{\mat}{\matrixsym}
\newcommand{\ten}{\tensorsym}
\DeclareMathOperator{\grad}{\nabla}
\DeclareMathOperator{\dive}{\text{div}}
\DeclareMathOperator{\curl}{\text{curl}}
\DeclareMathOperator{\tr}{\text{tr}}
\DeclareMathOperator{\sym}{\text{sym}}
\newtheorem{remark}{Remark}
\newtheorem{definition}{Definition}
\newcommand{\R}{\mathbb{R}}
\newcommand{\D}{\mathcal{D}}

\newcommand{\tin}{\text{in}}
\newcommand{\ton}{\text{on}}

\newtheorem{theorem}{Theorem}
\newtheorem{lemma}{Lemma}
\newcommand{\pts}[1]{[{\bf #1 puntos}] }

\begin{document}

\maketitle
\hfill \textbf{Fecha de entrega: 23:59 del 30/05/2025}
 
\todo[inline,color=white!90!black]{\textbf{Instrucciones: } La tarea debe ser entregada de manera individual en un informe en formato .pdf a través del buzón habilitado en la plataforma Canvas, donde deben mostrar también el código desarrollado. Para su conveniencia, pueden entregar las tareas en un Jupyter Notebook, de modo que sea más cómodo mostrar el código. En cualquier caso, se debe entregar un único archivo como respuesta a la tarea. La política de atrasos será: se calculará un factor lineal que vale 1 a la hora de entrega y 0 48 horas después. Esto multiplicará su puntaje obtenido. Pueden usar ChatGPT u otros modelos solo a conciencia. El uso de salidas de GPT sin su debida comprensión será severamente sancionado. }

\begin{enumerate}

    \item\pts{1} Dado un espacio de Hilbert $H$ considere una forma bilineal $a:H\times H\to \R$ acotada y elíptica. Demuestre que también satisface las hipótesis del Lema Generalizado de Lax-Milgram.

    \item\pts{3} Considere el problema de Stokes en $\Omega\subset \R^n$, dado por: Encuentre $\vec u$ en $H_0^1(\Omega,\R^n)$ y $p$ en $L_0^2(\Omega)$ tales que
            $$\begin{aligned}
                -\mu \Delta \vec u + \grad p &= \vec 0 &&\Omega \\
                \dive \vec u &=0 &&\Omega \\
                \vec u &= \vec 0 && \partial \Omega.
            \end{aligned}$$
            Muestre que este problema se puede escribir de forma mixta. Una formulación mixta consiste en encontrar $u$ en $V$ y $p$ en $Q$ tales que, dadas dos formas bilineales $a:V\times V\to \R$ y $b:V\times Q\to \R$ y funcionales lineales $F\in V'$ y $G\in Q'$, se tiene que
            $$\begin{aligned}
                a(u, v) + b(v, p) &= F(v) &&\forall v\in V\\
                b(u,q)            &= G(q) &&\forall q\in Q.
            \end{aligned}$$
            Esto se escribe en forma de operadores como 
            $$ \begin{bmatrix} A & B^T \\ B & 0 \end{bmatrix}\begin{bmatrix}u \\ p \end{bmatrix} = \begin{bmatrix} F \\ G \end{bmatrix}. $$
            Para el problema de Stokes, identifique cuales son los operadores $A$ y $B$, y averigüe por qué es importante considerar 
                $$L_0^2(\Omega) = \left\{ q \in L^2(\Omega): \int_\Omega q\,dx = 0 \right\}$$
            en lugar de simplemente el espacio $L^2(\Omega)$. 

    \item\pts{2} Condiciones suficientes para la existencia de soluciones de un problema mixto son que:
            \begin{itemize}
                \item $a$ sea una forma acotada y elíptica.
                \item $b$ sea continua y tal que satisface la condición inf-sup.
            \end{itemize}
            Muestre que bajo estas hipótesis, el problema de Stokes en forma monolítica, i.e. escrito en $H = H_0^1(\Omega, \R^n) \times L_0^2(\Omega)$ como
                $$ M((u,p), (v,q)) = \ell((v,q)) \qquad\forall (v,q) \in H $$
                donde 
                $$\begin{aligned}
                    M((u,p), (v,q)) &:= a(u,v) + b(u,q) + b(v,p) \\
                    \ell((v,q))  &:= F(v) + G(q),
                \end{aligned}$$
                satisface una condición inf-sup. 

    \item Considere el problema ADR, dado por encontrar $u$ en $H_0^1(\Omega)$ tal que
            $$ -\mu \Delta u + \vec b \cdot \grad u + cu = f, $$
            para alguna $f$ en $H^{-1}(\Omega)$. 
            \begin{itemize}
                \item\pts{1} Proponga un espacio discreto de elementos finitos que sea conforme en $H^1$ y que permita aproximar a la solución continua del problema ADR.
                \item\pts{1} Demuestre que el problema discreto está bien puesto. 
                \item\pts{2} Demuestre que esta elección de espacio genera una aproximación de la solución que es convergente a la solución. Muestre además la tasa de convergencia teórica esperada para el esquema discreto propuesto. 
            \end{itemize}

    \item Considere $\Omega = (0,1)^2$. Implemente un código de elementos finitos que aproxime el problema ADR con condición de Dirichlet $u=1$ en el lado derecho y Neumann homogéneo en el resto, puede elegir si aproximar con triángulos o con cuadrados. Para ello: 
            \begin{itemize}
                \item\pts{1} Dado un número de intervalos por lado, implemente las matrices \texttt{IEN} y \texttt{coords}. Muestre su resultado para un cuadrado con dos elementos por lado, y grafique la geometría discretizada con los elementos enumerados para validar el resultado.
                \item\pts{3} Considere el elemento de referencia y un elemento global igual al de referencia por simplicidad ($K = \hat K$). Calcule con una regla de cuadratura adecuada la matriz local asociada a cada uno de los operadores del problema y valide los valores numéricos obtenidos con las expresiones analíticas de las integrales: 
                    \begin{itemize}
                        \item $-\Delta u$
                        \item $\vec b \cdot \grad u$
                        \item u
                    \end{itemize}
                \item\pts{2} Repita el punto anterior para el vector local con un lado derecho dado por $f(x) = 1$. 
                \item\pts{3} A partir de las rutinas anteriores, genere una función \texttt{getProblem(mu,b,c,f)} donde $\mu,b,c$ son escalaes y $f:\R\to\R$ que entregue la matriz $\mat A_h$ que define el problema y el lado derecho $\vec F_h$.
                \item\pts{2} Explique cómo implementar las condiciones de borde y genere una función que modifique el problema $(\mat A_h, \vec F_h)$ para imponer la condición de Dirichlet dada ($u=1$ en el lado derecho).
                \item\pts{3} Usando el método de soluciones manufacturadas, muestre que la solución discreta converge a la continua con la tasa esperada.
            \end{itemize}
\end{enumerate}

\todo[inline,color=white!90!black]{\textbf{Nota: } Abriremos un foro en Canvas para revisar cualquier typo y/o error que haya en el enunciado.}
\end{document}

