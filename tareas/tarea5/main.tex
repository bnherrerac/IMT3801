\documentclass{article}
\usepackage[utf8]{inputenc}
\usepackage{amsmath, amsthm, amssymb, mathpazo, isomath, mathtools}
\usepackage{subcaption,graphicx,pgfplots}
\usepackage{fullpage}
\usepackage{booktabs}
\usepackage{hyperref}
\usepackage{algorithm, algorithmic}
\usepackage{mathtools}
\usepackage{todonotes}

\title{Tarea 5}
%\author{Nicol\'as A Barnafi\thanks{Instituto de Ingeniería Biológica y Médica, Pontificia Universidad Católica de Chile, Chile}, Axel Osses\thanks{Departamento de Ingeniería Matemática, Universidad de Chile, Chile}}
%\author{Nicol\'as A Barnafi}
\date{}

\renewcommand{\vec}{\vectorsym}
\newcommand{\mat}{\matrixsym}
\newcommand{\ten}{\tensorsym}
\DeclareMathOperator{\grad}{\nabla}
\DeclareMathOperator{\dive}{\text{div}}
\DeclareMathOperator{\curl}{\text{curl}}
\DeclareMathOperator{\tr}{\text{tr}}
\DeclareMathOperator{\sym}{\text{sym}}
\newtheorem{remark}{Remark}
\newtheorem{definition}{Definition}
\newcommand{\R}{\mathbb{R}}
\newcommand{\D}{\mathcal{D}}
\newcommand{\parder}[2]{\frac{\partial\,#1}{\partial\,#2}}

\newcommand{\tin}{\text{in}}
\newcommand{\ton}{\text{on}}

\newtheorem{theorem}{Theorem}
\newtheorem{lemma}{Lemma}
\newcommand{\pts}[1]{[{\bf #1 puntos}] }

\begin{document}

\maketitle
\hfill \textbf{Fecha de entrega: 23:59 del 13/06/2025}
 
\todo[inline,color=white!90!black]{\textbf{Instrucciones: } La tarea debe ser entregada de manera individual en un informe en formato .pdf a través del buzón habilitado en la plataforma Canvas, donde deben mostrar también el código desarrollado. Para su conveniencia, pueden entregar las tareas en un Jupyter Notebook, de modo que sea más cómodo mostrar el código. En cualquier caso, se debe entregar un único archivo como respuesta a la tarea. La política de atrasos será: se calculará un factor lineal que vale 1 a la hora de entrega y 0 48 horas después. Esto multiplicará su puntaje obtenido. Pueden usar ChatGPT u otros modelos solo a conciencia. El uso de salidas de GPT sin su debida comprensión será severamente sancionado. }

\begin{enumerate}
    \item Considere la deformación dada por 
        \begin{align*}
            x_1(\vec X, t) &= e^tX_1 - e^{-t} X_2 \\
            x_2(\vec X, t) &= e^t X_1 + e^{-t}X_2 \\
            x_3(\vec X, t) &= X_3. 
        \end{align*}
    \begin{itemize}
        \item\pts{1} Calcule el campo de desplazamiento, la velocidad, y la aceleración en coordenadas materiales (de referencia). 
        \item\pts{1} Calcule el campo de desplazamiento, la velocidad y la aceleración en coordenadas espaciales (deformadas). 
        \item\pts{1} Calcule los tensores $\ten F$, $\ten C$, y $\ten E$.
        \item\pts{1} Ignore la tercera componente y considere $\Omega$ un cuadrado unitario. Grafique en Python la configuración inicial ($t=0$), y la configuración deformada en los instantes 0.5, 1.0 y 2.0. 
        \item\pts{1} En los instantes descritos, grafique la norma de Frobenius de $\ten C$ en configuración de referencia y en la deformada. Comente cómo cambia la interpretación del tensor entre ambas configuraciones, y proponga una manera de modificar el tensor $\ten C$ para que tenga una mejor interpretación en la configuración deformada.
    \end{itemize}
    \begin{itemize}
        \item\pts{1} Dada una matriz $\mat A$ se definen 
            $$I_1(\mat A) = \tr \mat A,$$ 
            $$I_2(\mat A) = \tr \left(\det (\mat A) \mat A^{-T}\right) = \frac 1 2 \left([\tr \mat A]^2 - \tr(\mat A^2)\right),$$ 
            $$I_3(\mat A) = \det A.$$ 

            Muestre que si se hace un cambio de base a la matriz $\mat A$ en el sentido de reemplazarla por $\mat Q^{-1} \mat A \mat Q$, estas funciones no cambian su valor, i.e. $I_i(\mat A) = I_i(\mat Q^{-1} \mat A\mat Q)$. Por esta razón se les llama \emph{invariantes} de $\mat A$. Hint: Para el determinante podría ser útil investigar sobre el símbolo de Levi-Civita $\epsilon_{ijk}$ y su uso en el cálculo del determinante. 
        \item\pts{1} El Teorema de Caley-Hamilton establece que toda matriz satisface su propia ecuación característica usando las invariantes: 
                $$ \mat A^3 - I_1(\mat A) \mat A^2 + I_2(\mat A) \mat A - I_3(\mat A) \mat I = 0. $$
                Derive esta ecuación con respecto a $\mat A$ y desarrolle la ecuación para demostrar que $\parder{\det \mat A}{\mat A} = \det (\mat A) \mat A^{-T}$.
    \end{itemize}
    \item Considere un material hiperelástico Neo-Hookeano, donde la energía hiperelástica está dada por
            $$ \Psi(\ten F) = \frac C 2(\tr \ten C - 3) + \frac \kappa 2 \left( J -1\right)^2. $$
            \begin{itemize}
                \item\pts{1} Calcule el tensor de Piola.
                \item\pts{1} Escriba la formulación débil asociada a la ecuación de momentum con el tensor explícito de Piola calculado.
            \end{itemize}
    \item Modifique la deducción de la conservación de masa para considerar un flujo de masa dado por su gradiente normal, i.e. considere un término adicional de superficie para todo subdominio $\omega_t\subseteq\Omega_t$: 
            $$ \int_{\partial \omega_t} \ten K \grad\rho \cdot \vec n \,dS, $$
            \begin{itemize}
                \item\pts{1} Muestre la ecuación resultante, y muestre que el operador diferencial en espacio es un operador ADR. 
            \end{itemize}
    \item Considerar ecuación de momentum lineal de referencia: 
            $$ \rho_0\ddot{\vec u} - \dive_X \ten P(\ten F)=0. $$
            \begin{itemize}
                \item\pts{1}  Cambie de variables el tensor $\ten P$ para que dependa de $\ten E$, y luego linealice con respecto a $\vec u = 0$. Recuerde que linealizar una ecuación
                    $$ F(x) = 0 $$
                    con respecto a un punto $x_0$ significa reemplazar $F(x) \approx F(x_0) + dF(x_0)[\delta x]$ para obtener una ecuación para el incremento $\delta x$. Acá, $dF(x_0)[\delta x]$ es la derivada de Gateaux en dirección $\delta x$. 
                \item\pts{1} Muestre que el problema resultante es el problema de elastodinámica lineal, e identifique claramente el tensor de Hooke que obtiene. 
            \end{itemize}

    \item\pts{1} Defina su proyecto de curso. 
\end{enumerate}

\todo[inline,color=white!90!black]{\textbf{Nota: } Abriremos un foro en Canvas para revisar cualquier typo y/o error que haya en el enunciado.}
\end{document}

