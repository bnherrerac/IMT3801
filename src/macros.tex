\usepackage[utf8]{inputenc}
\usepackage{amsmath, amsthm, amssymb, mathpazo, isomath, mathtools}
\usepackage{subcaption,graphicx}
\usepackage{fullpage}
\usepackage{booktabs}
\usepackage{hyperref}
%%%% Remove some annoying hyperref warnings. %%%%%%%%%%%%%%%%%%%%%%%%%%%%%%%%%%%%%%%%%%%%%%%%%%%%%%%%%%%
% See https://tex.stackexchange.com/questions/10555/hyperref-warning-token-not-allowed-in-a-pdf-string
\makeatletter
\pdfstringdefDisableCommands{\let\HyPsd@CatcodeWarning\@gobble}
\makeatother
%%%%%%%%%%%%%%%%%%%%%%%%%%%%%%%%%%%%%%%%%%%%%%%%%%%%%%%%%%%%%%%%%%%%%%%%%%%%%%%%%%%%%%%%%%%%%%%%%%%%%%%%

\usepackage{algorithm}
\usepackage{algpseudocode}
\usepackage{mathtools}
\usepackage{todonotes}
\usepackage{cancel}
\usepackage{pgfplots}
\pgfplotsset{compat=1.18}
\newcommand{\RightComment}[1]{\hfill \(\triangleright\) \textit{#1}}
\newcommand{\example}[1]{\todo[inline,color=orange!10!white]{\textbf{Example:} #1}}
\newcommand{\nota}[1]{{\colorbox{red!30}{[#1]}}}

\renewcommand{\vec}{\vectorsym}
\newcommand{\mat}{\matrixsym}
\newcommand{\ten}{\tensorsym}
\DeclareMathOperator{\grad}{\nabla}
\DeclareMathOperator{\Grad}{\text{Grad}}
\DeclareMathOperator{\Cof}{\text{Cof}}
\DeclareMathOperator{\dive}{\text{div}}
\DeclareMathOperator{\Dive}{\text{Div}}
\DeclareMathOperator{\curl}{\text{curl}}
\DeclareMathOperator{\Curl}{\text{Curl}}
\DeclareMathOperator{\tr}{\text{tr}}
\newcommand{\parder}[2]{\frac{\partial #1}{\partial #2} }

\newtheorem{remark}{Remark}
\newtheorem{definition}{Definition}
\newtheorem{note}{Note}

\newcommand{\R}{\mathbb{R}}
\newcommand{\D}{\mathcal{D}}
\newcommand{\T}{\mathcal{T}}
\newcommand{\tenF}{\ten{F}}
\newcommand{\vX}{\nabla_X}
\newcommand{\vx}{\nabla_x}
\newcommand{\vvarphi}{\vec{\varphi}}
\renewcommand{\P}{\mathcal{P}}

\newcommand{\tin}{\text{in }}
\newcommand{\ton}{\text{on }}

\newtheorem{theorem}{Theorem}
\newtheorem{lemma}{Lemma}

\usepackage{listings}
\usepackage{xcolor}
\definecolor{codegreen}{rgb}{0,0.6,0}
\definecolor{codegray}{rgb}{0.5,0.5,0.5}
\definecolor{codepurple}{rgb}{0.58,0,0.82}
\definecolor{backcolour}{rgb}{0.95,0.95,0.92}
\lstdefinestyle{mystyle}{
  backgroundcolor=\color{backcolour}, commentstyle=\color{codegreen},
  keywordstyle=\color{magenta},
  numberstyle=\tiny\color{codegray},
  stringstyle=\color{codepurple},
  basicstyle=\ttfamily\footnotesize,
  breakatwhitespace=false,         
%   breaklines=false,                     
  captionpos=b,                    
  keepspaces=true,                 
  numbers=none,                    
  numbersep=5pt,                  
  showspaces=false,                
  showstringspaces=false,
  showtabs=false,                  
  tabsize=2,
%   frameround=tttn,
  framerule=1.5pt,
  rulecolor=\color{red!60!black}
}
\lstset{style=mystyle}

