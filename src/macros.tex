\usepackage[utf8]{inputenc}
\usepackage{amsmath, amsthm, amssymb, mathpazo, isomath, mathtools}
\usepackage{thmtools}
\usepackage{subcaption,graphicx}
\usepackage{fullpage}
\usepackage{booktabs}
\usepackage{hyperref}
\usepackage{environ}
%%%% Remove some annoying hyperref warnings. %%%%%%%%%%%%%%%%%%%%%%%%%%%%%%%%%%%%%%%%%%%%%%%%%%%%%%%%%%%
% See https://tex.stackexchange.com/questions/10555/hyperref-warning-token-not-allowed-in-a-pdf-string
\makeatletter
\pdfstringdefDisableCommands{\let\HyPsd@CatcodeWarning\@gobble}
\makeatother
%%%%%%%%%%%%%%%%%%%%%%%%%%%%%%%%%%%%%%%%%%%%%%%%%%%%%%%%%%%%%%%%%%%%%%%%%%%%%%%%%%%%%%%%%%%%%%%%%%%%%%%%

\usepackage{algorithm}
\usepackage{algpseudocode}
\usepackage{mathtools}
\usepackage{todonotes}
\usepackage{cancel}
\usepackage{pgfplots}
\pgfplotsset{compat=1.18}

\DeclareMathOperator{\spanned}{\text{span}}
\DeclareMathOperator{\grad}{\nabla}
\DeclareMathOperator{\Grad}{\text{Grad}}
\DeclareMathOperator{\Cof}{\text{Cof}}
\DeclareMathOperator{\dive}{\text{div}}
\DeclareMathOperator{\Dive}{\text{Div}}
\DeclareMathOperator{\curl}{\text{curl}}
\DeclareMathOperator{\Curl}{\text{Curl}}
\DeclareMathOperator{\tr}{\text{tr}}
\DeclareMathOperator{\inv}{\text{Inv}}
\DeclareMathOperator{\dom}{\text{dom}}
\DeclareMathOperator{\supp}{\text{supp}}
\DeclareMathOperator{\diag}{\text{diag}}
\DeclareMathOperator{\bolddiag}{\textbf{diag}}
\DeclareMathOperator{\im}{\text{im}}
\DeclareMathOperator{\dist}{\text{dist}}
\DeclareMathOperator{\rank}{\text{rank}}
\DeclareMathOperator{\interior}{\text{int}}


\declaretheoremstyle[
  spaceabove=6pt, spacebelow=12pt,
  headfont=\normalfont\bfseries,
  notefont=\mdseries, notebraces={(}{)},
  bodyfont=\normalfont,
  postheadspace=1em,
]{thmstyle}
\declaretheoremstyle[
  spaceabove=6pt, spacebelow=12pt,
  headfont=\normalfont\bfseries,
  notefont=\mdseries, notebraces={(}{)},
  bodyfont=\normalfont,
  postheadspace=1em,
]{lemmastyle}
\declaretheoremstyle[
  spaceabove=6pt, spacebelow=12pt,
  headfont=\normalfont\bfseries,
  notefont=\mdseries, notebraces={(}{)},
  bodyfont=\normalfont,
  postheadspace= 0.5em,
]{defstyle}
\declaretheoremstyle[
  spaceabove=6pt, spacebelow=12pt,
  headfont=\normalfont\bfseries,
  notefont=\mdseries, notebraces={(}{)},
  bodyfont=\normalfont,
  postheadspace=1em,
]{smallstyle}

\declaretheorem{theorem}[style=thmstyle, parent=chapter, name=Theorem]
\declaretheorem{lemma}[style=thmstyle, sibling=theorem, name=Lemma]
\declaretheorem{remark}[style=smallstyle, sibling=theorem, name=Remark]
\declaretheorem{definition}[style=defstyle, sibling=theorem, name=Definition]
\declaretheorem{corollary}[style=smallstyle, sibling=theorem, name=Corollary]
\declaretheorem{note}[style=smallstyle, sibling=theorem, name=Note]

\newcommand{\R}{\mathbb{R}}
\newcommand{\C}{\mathbb{C}}
\newcommand{\D}{\mathcal{D}}
\newcommand{\T}{\mathcal{T}}
\newcommand{\N}{\mathbb{N}}
\newcommand{\tenF}{\ten{F}}
\newcommand{\vX}{\nabla_X}
\newcommand{\vx}{\nabla_x}
\newcommand{\vvarphi}{\vec{\varphi}}
\newcommand{\RightComment}[1]{\hfill \(\triangleright\) \textit{#1}}
\newcommand{\example}[1]{\todo[inline,color=orange!10!white]{\textbf{Example:} #1}}
\newcommand{\nota}[1]{{\colorbox{red!30}{[#1]}}}

\newcommand{\mat}{\matrixsym}
\newcommand{\ten}{\tensorsym}
\newcommand{\parder}[2]{\frac{\partial #1}{\partial #2} }
\newcommand{\tin}{\text{ in }}
\newcommand{\ton}{\text{ on }}

\renewcommand{\P}{\mathcal{P}}
\renewcommand{\vec}{\vectorsym}

\usepackage{listings}
\usepackage{xcolor}
\definecolor{codegreen}{rgb}{0,0.6,0}
\definecolor{codegray}{rgb}{0.5,0.5,0.5}
\definecolor{codepurple}{rgb}{0.58,0,0.82}
\definecolor{backcolour}{rgb}{0.95,0.95,0.92}
\lstdefinestyle{mystyle}{
  backgroundcolor=\color{backcolour}, commentstyle=\color{codegreen},
  keywordstyle=\color{magenta},
  numberstyle=\tiny\color{codegray},
  stringstyle=\color{codepurple},
  basicstyle=\ttfamily\footnotesize,
  breakatwhitespace=false,         
%   breaklines=false,                     
  captionpos=b,                    
  keepspaces=true,                 
  numbers=none,                    
  numbersep=6pt,                  
  showspaces=false,                
  showstringspaces=false,
  showtabs=false,                  
  tabsize=2,
%   frameround=tttn,
  framerule=1.6pt,
  rulecolor=\color{red!60!black}
}
\lstset{style=mystyle}

\NewEnviron{tightalign*}{%
  \setlength{\abovedisplayskip}{6pt}%
  \setlength{\belowdisplayskip}{6pt}%
  \setlength{\abovedisplayshortskip}{6pt}%
  \setlength{\belowdisplayshortskip}{6pt}%
  \begin{align*}
    \BODY % \BODY is the content of the environment
  \end{align*}
}

\NewEnviron{tightalign}{%
  \setlength{\abovedisplayskip}{6pt}%
  \setlength{\belowdisplayskip}{6pt}%
  \setlength{\abovedisplayshortskip}{6pt}%
  \setlength{\belowdisplayshortskip}{6pt}%
  \begin{align}
    \BODY % \BODY is the content of the environment
  \end{align}
}