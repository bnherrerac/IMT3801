Instead of discretizing the differential operator, as one would do in the case of finite differences to obtain discrete derivatives, one can consider discrete functional spaces, with the idea that the discrete space somehow converges to the continuous space. This is known as a Galerkin scheme.  
\section{Galerkin schemes}\label{sec:galerkin-schemes}
Consider thus an abstract differential problem given by finding $u$ in $V$ such that
\begin{equation}
    a(u, v) = L(v) \qquad \forall v \in V,
\end{equation}
such that the hypotheses the Lax-Milgram lemma~\ref{lemma:lax-milgram} hold. In this case, we can consider a discrete space $V_h$ that approximates $V$, and define a Galerkin scheme as the following discrete problem: find $u_h \in V_h$ such that 
\begin{equation}\label{eq:galerkinscheme}
    a(u_h, v_h) = L(v_h) \qquad \forall v_h \in V_h.
\end{equation}
The most notable aspect of Lax-Milgram is that all of its hypotheses hold also in $V_h$, which implies that the Galerkin scheme is also invertible, and the \emph{a priori} estimate holds as well. The natural question is whether the discrete solution $u_h$ converges to the continuous solution $u$, which is studied through the \emph{error equation}. This is computed by setting the continuous test function $V$ as $V_h$ and then subtracting both problems: 
\begin{equation}
    a(e_h, v_h) = 0 \qquad \forall v_h \in V_h,
\end{equation}
where $e_h = u - u_h$. This property is known as \emph{Galerkin orthogonality}, and it can be used to compute the error estimate by considering an arbitrary function $z_h$ in $V_h$:
\begin{tightalign*}
    \alpha \| e_h \|_V^2 &\leq a(e_h, e_h) \\ 
    &= a(e_h, u - z_h) \tag{Galerkin orthogonality} \\
    &\leq C\|e_h\|_V \|u - z_h\|_V \tag{$a$ continuous},
\end{tightalign*}
where one obtains that for all $z_h$ it holds that
\begin{equation}
    \| e_h \|_V \leq \frac C \alpha \|u - z_h\|_V.
\end{equation}
Taking the infimum over $z_h$ one obtains the celebrated \emph{Ceà estimate}: 
\begin{equation}
    \| u - u_h \|_V \leq \frac C \alpha \inf_{v_h\in V_h} \|u - v_h\|_{V_h}.
\end{equation}
This inequality can reveal many things. For example, if the number $C/\alpha$ is very big, it can hint on a very wide gap between the optimal solution (i.e. the projection) and the discrete one computed from the space $V_h$. A more precise characterization of the approximation properties of a space can be given by the \emph{Kolmogorov width}, which has been studied in~\cite{evans2009n}.

\todo[inline]{NB: Esto seguro va en FEM}
\section{Inverse inequalities}\label{thm:inverse-inequalities}
All of the discrete spaces considered are finite dimensional, which means that all their norms are equivalent. Of course, these relationships don't hold in the continuous setting, so there is bound to be some dependence of these constants on $h$. The great thing about inverse inequalities is that in the discrete settings we can sometimes get away with operations that would be otherwise not feasible, but using adequate bounds can yield convergence anyway. We provide a general result first and then show some important consequences. For details, see~\cite{ern2004theory}. We will require the following definition: we say that a family of affine meshes in $\R^d$ is \emph{shape regular} if there exists $\sigma_0$ such that 
\begin{equation}
    \forall h: \sigma_K\coloneqq \frac{h_K}{\rho_K} \leq \sigma_0 \quad\forall K\in \T_h,
\end{equation}
where $\rho_K$ is the diameter of the largest ball that can be inscribed in $K$, and $h_K$ is the diameter of $K$. It is \emph{quasi-uniform} if it is shape-regular and there is some $C$ such that
\begin{equation}\label{eq:def-quasi-uniform}
    \forall h: h_K \geq C h \quad \forall K\in \T_h.
\end{equation}
\begin{theorem}[Global inverse inequality]\label{thm:global-inverse-inequality}
    Consider a finite element $(\hat K, \hat P, \hat \Sigma)$, $l\geq 0$ such that $\hat P\subset W^{l,\infty}(\hat K)$, a family of shape-regular and quasi-uniform meshes $\{\T_h\}_{h>0}$ with $h<1$, and set
    \begin{equation}
        W_h = \{v_h: v_h\circ T_K \in \hat P \qquad\forall K\in \T_h \},
    \end{equation}
    where $T_K$ is the affine mapping from an element $K$ to the reference element $\hat K$; i.e. the family of discrete functions that locally belong to the finite element considered. Then, there is some positive $C$ such that for all $v_h$ in $W_h$ and $m\in [0,l]$ it holds that
    \begin{equation}
        \left(\sum_K \| v_h\|^p_{W^{l,p}(K)}\right)^{1/p} \leq Ch^{m-l+\min(0, d/p - d/q)} \left(\sum_K \|v_h\|_{W^{m,q}(K)}^q\right)^{1/q}.
    \end{equation}
\end{theorem}

In particular, the above inequality shows that
\begin{equation*}
    \| v_h \|_{W^{1,p}} \leq C h^{-1} \| v_h \|_{L^p}.
\end{equation*}
One can also obtain the following local estimate in 2D for simplices~\cite{warburton2003constants}:
\begin{equation}
    \| \vec v_h \|_{0,F} \leq C h_K^{-1/2} \| \vec v_h \|_{0,K},
\end{equation}
where $F$ is a facet (line or triangle) and $K$ is the element (triangle or tetrahedron).

\todo[inline]{NB: Por mover}
\section{Conforming schemes}\label{sec:conforming-schemes}
Galerkin schemes are based on discrete approximation spaces, which may or may not be contained in the underlying function space.
\begin{definition}[Conforming scheme]\label{def:conforming-scheme}
    Let $V$ be an infinite-dimensional space, such as a Sobolev space, and $V_h$ a finite-dimensional approximation space associated with a discrete scheme. We say that this scheme is \emph{conforming} if $V_h\subset V$, and \emph{non-conforming} if $V_h\not\subset V$. 
\end{definition}
% \begin{lemma}[Conforming spaces]
%     Consider two non-overlapping Lipschitz domains $K_1$ and $K_2$ such that they meet at a common surface $\Sigma$. 
%     \begin{itemize}
%         \item Consider two scalar functions $p_1$ in $H^1(K_1)$ and $p_2$ in $H^1(K_2)$, and glue them as $p = p_1 I_{K_1} + p_2 I_{K_2}$. If $p_1|_\Sigma = p_2|_\Sigma$, then $p$ belongs to $H^1(K_1\cup K_2\cup \Sigma)$. 
%         \item Consider two vector functions $\vec u_1$ in $H(\dive;K_1)$ and $\vec u_2$ in $H(\dive; K_2)$, and glue them as $\vec u = \vec u_1 I_{K_1} + \vec u_2 I_{K_2}$. Then, if $\vec u_1\cdot \vec n= \vec u_2\cdot \vec n$ it holds that $\vec u$ belongs to $H(\dive; K_1\cup K_2 \cup \Sigma)$. 
%         \item Consider two vector functions $\vec u_1$ in $H(\curl;K_1)$ and $\vec u_2$ in $H(\curl; K_2)$, and glue them as $\vec u = \vec u_1 I_{K_1} + \vec u_2 I_{K_2}$. Then, if $\vec u_1\times \vec n= \vec u_2\times \vec n$ it holds that $\vec u$ belongs to $H(\curl; K_1\cup K_2 \cup \Sigma)$. 
%     \end{itemize}
%     \begin{proof}
%         Point (1) is proved in \cite{gatica2014simple}, (2) is in \cite{monk2003finite}, and (3) is homework :) . 
%     \end{proof}
% \end{lemma}


\section{Non-conforming schemes}\label{sec:non-conforming-schemes}
In order to have a good approximation, it is not necessary that the scheme is conforming. Some examples of conforming schemes are most finite element methods and spectral element methods, and among non-conforming schemes are discontinuous-Galerkin (DG) finite element methods, and methods that impose boundary conditions weakly. We will study several conforming finite element methods in Chapter~\ref{chapter:fem}. 

Although conforming spaces have some nice properties, there exist some applications where the mesh and the domain boundaries may not match, or where traditional finite elements may not apply, forcing one to use other schemes such as spline-based methods, where the degrees of freedom are control points of the splines, rather than actual nodes. In these cases, we can use a non-conforming scheme. The following presentation is based on~\cite{Chouly2024}. Consider the Poisson problem with a nonhomogeneous Dirichlet boundary condition: find $u:\Omega\to \mathbb{R}$ such that
\begin{equation}
    \begin{aligned}
        -\Delta u &= f &&\tin \Omega \\
        \gamma_0 u &= g &&\ton \partial\Omega,
    \end{aligned}
\end{equation}
which we rewrite in weak form as follows: find $u\in H^1(\Omega)$ such that $u|_{\partial\Omega} = g$ and
\begin{equation}
    a(u,v) = (f,v) \qquad \forall v\in H_0^1(\Omega),
\end{equation}
where as usual we denote $a(u,v)=(\nabla u,\nabla v)$. Let $K^h$ be a discretization of $\Omega$ with mesh size $h$, which we assume is sufficiently regular. There exist several ways to enforce the Dirichlet boundary condition, such as the \emph{penalty method} and the \emph{Nitsche method}. We outline both methods below.
\begin{itemize}
    \item \emph{The penalty method}: at the continuous level, this method can be formulated as follows: find $u^\varepsilon\in H^1(\Omega)$ such that 
    \begin{equation}
        a(u^\varepsilon, v) + \frac{1}{\varepsilon} (u^\varepsilon, v)_{\partial\Omega} = (f,v) + \frac{1}{\varepsilon} (g,v)_{\partial\Omega} \qquad \forall v\in H^1(\Omega),
    \end{equation}
    where we introduced the penalty parameter $\varepsilon>0$. When going back to the strong form, we verify that $u^\varepsilon$ satisfies the Poisson equation $-\Delta u^\varepsilon = f$ and the Robin boundary condition 
    \begin{equation}
        \nabla u^\varepsilon \cdot\vec n = -\frac{1}{\varepsilon}(u^\varepsilon - g) \implies \varepsilon (\nabla u^\varepsilon) \cdot \vec n = -(u^\varepsilon - g),
    \end{equation}
    which for $\varepsilon$ small enough approximates the nonhomogeneous Dirichlet boundary condition. By the Friedrich inequality, we can show that the bilinear form in the left hand side is elliptic on $H^1(\Omega)$ and thus the problem is well-posed by the Lax-Milgram lemma. In a discrete setting, we consider $\varepsilon = \varepsilon_0 h^\lambda$ for some $\varepsilon_0 > 0$ and $\lambda\geq 0$, both independent of the mesh, and we can prove that this discrete problem is well-posed and convergent. Often, the user has to manually tune the values of $\varepsilon_0$ and $\lambda$ to achieve good convergence rates. The critical choice lies in the value of $\varepsilon_0$: if the value is too small, the conditioning of the global stiffness matrix deteriorates, since its conditioning is $\mathcal{O}(\varepsilon_0^{-1}h^{-1-\lambda})$, and if the value is too large, the Dirichlet condition is approximated poorly. 

    \item \emph{The Nitsche method}: let $\gamma > 0$ be a positive function on $\partial\Omega$ and $\theta\in\mathbb{R}$ a fixed parameter. Integrating by parts the weak form of the Poisson problem we first get 
    \begin{equation}
        a(u,v) - (\nabla u\cdot \vec n, v)_{\partial\Omega} = (f,v),
    \end{equation}
    and from the Dirichlet condition we can write 
    \begin{equation}
        (u,\gamma v -\theta\nabla v\cdot \vec n)_{\partial\Omega} = (g,\gamma v - \theta \nabla v \cdot \vec n)_{\partial\Omega}.
    \end{equation}
    Adding these two equations together and rearranging, we get 
    \begin{equation}
        a(u,v) - (\nabla u \cdot \vec n, v)_{\partial\Omega} - \theta (u,\nabla v\cdot  \vec n) + (u,\gamma v)_{\partial\Omega} = (f,v) + (g,\gamma v - \theta \nabla v\cdot n)_{\partial\Omega}.
    \end{equation}
    Let $\zeta$ denote a piecewise constant function on the boundary, that is defined locally by the value of the diameter of every boundary facet. Taking $\gamma = \gamma_0 \zeta^{-1}$ for some $\gamma_0>0$, and recalling the trace inequality~\eqref{eq:trace-inequality},
    \begin{equation}
        \|\nabla v_h\cdot \vec n\|^2_{-1/2,\partial\Omega} \leq c_T \|\nabla v_h\|^2_{0,\Omega},
    \end{equation}
    we can prove that this problem is well-posed provided that 
    \begin{equation}
        \frac{(1+\theta)^2 c_T}{\gamma_0}\leq 1.
    \end{equation}
    Moreover, this method is convergent in the $H^1$ norm for large enough $\gamma_0$. In the case that we expect more regularity, for $u\in H^s(\Omega)$ with $3/2<s<1+k$ (where $k$ is the degree of the polynomial approximation space), we get
    \begin{equation}
        \|u-u_h\| + \|\nabla u\cdot \vec n - \nabla u_h\cdot \vec n\|_{-1/2,\partial\Omega} \leq Ch^s\|u\|_{s,\Omega}.
    \end{equation}
    Remarkably, and in contrast to the penalty method, the constant $C>0$ does not depend on $\gamma_0$ provided that it is large enough, but does depend on the regularity of the mesh and on the polynomial order $k$. As expected, the value of $\gamma_0$ influences the condition number of the global stiffness matrix associated to the left hand side of this problem, and thus it must not be taken too large, but the impact of the value of $\gamma_0$ on the approximation of the Dirichlet boundary condition is much smaller than in the penalty method. 
\end{itemize}

In practice, the penalty method is much simpler to understand and to implement, but its accuracy in some specific problems may not always be satisfactory. The Nitsche method is still simple to implement, and it constitutes a better alternative to the penalty method, where one has to tune only one numerical parameter. There exist more variants to these methods, such as the penalty-free Nitsche method and methods with Lagrange multipliers. The interested reader is referred to~\cite{Chouly2024} for more details.