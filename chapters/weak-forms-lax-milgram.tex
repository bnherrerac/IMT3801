\section{Weak formulations}
A weak formulation refers to an integral form of a PDE, understood distributionally. This is typically a systematic procedure that should not be too difficult, and it helps in revealing what are the adequate boundary conditions for a given problem. The main tool for this will be the integration by parts formulas. Our test problem will be the Laplace problem, given by the $-\Delta$ operator. The minus sign will be better justified in the following section. Consider then the problem of finding $u$ such that 
\begin{equation}
    -\Delta u = f \qquad \text{ in $\Omega$}.
\end{equation}
Define an arbitrary smooth function $v$, then integration by parts yields
\begin{equation}\label{eq:ibp-poisson}
    - \int_\Omega \Delta u v\,dx = -\int_{\partial\Omega}\gamma_D v \gamma_N \grad u \,ds + \int_\Omega \grad u \cdot \grad v\,dx
\end{equation}
for all $v$. This function is typically called a \emph{test function}. The surface form suggest the boundary conditions: 
\begin{equation}
    \int_{\partial\Omega}\underbrace{\gamma_D v}_\text{Dirichlet BC} \underbrace{\gamma_N \grad u}_\text{Neumann BC} \,ds,
\end{equation}
so that we can have boundary conditions on the function itself  
\begin{equation*}
    u = g,
\end{equation*}
or on its normal derivative
\begin{equation*}
    \grad u \cdot \vec n = h.
\end{equation*}
This can be combined, so that for a given partition of the boundary into two disjoint sets $\Gamma_D$ and $\Gamma_N$ such that $\overline{\partial\Omega} = \overline{\Gamma_D}\cup\overline{\Gamma_N}$, one can have a Dirichlet boundary condition on $\Gamma_D$ and a Neumann boundary condition on $\Gamma_N$. For this type of boundary condition, one must define a solution space given by 
\begin{equation}
    V_g = \{v \in H^1(\Omega): \gamma_D v = g \text{ on $\Gamma_D$}\},
\end{equation}
but let us focus first on spaces with null Dirichlet boundary condition ($V_0$). In this case, the boundary conditions will give
\begin{equation*}
    \int_{\partial\Omega}\gamma_D v \gamma_N \grad u\,ds = \int_{\gamma_N} h \gamma_D v\,ds,
\end{equation*}
and thus the integral form of the equation will be given by
\begin{equation*}
    -\int_\Omega\Delta u v\,dx = -\int_{\Gamma_N}v h\,ds + \int_\Omega \grad u \cdot \grad v\,dx = \int_\Omega f v\,dx,
\end{equation*}
for all smooth $v$. Now, we note that (i) this formulation is well defined for $u,v$ in $H^1$ (and thus can be extended to hold for all $v$ in $H^1$ by density as long as $v$ satisfies the Dirichlet boundary conditions), (ii) the $\gamma_D$ operator has been omitted from the surface integral for convenience, and (iii) that the Dirichlet boundary condition does not appear anywhere in the formulation. This justifies naming Dirichlet boundary conditions \emph{essential}, and Neumann boundary conditions \emph{natural}. The \emph{weak formulation} of the problem thus refers to the following statement: Find $u$ in $V_0$ such that
\begin{equation}\label{eq:weak-form-poisson}
    (\grad u, \grad v)_{0,\Omega} = \langle f, v\rangle  + \langle h,  v\rangle \qquad \forall v\in V_0,
\end{equation}
for given functions $f$ in $V_0'$ and $g$ in $(\gamma_D V_0)'$. Note the following: 
\begin{itemize}
    \item The space of the solution and the test functions is the same. This is not mandatory, but it is common and can be better motivated by interpreting the Laplace problem as the first order equations related to the following minimization problem: 
    \begin{equation*}
        \min_{v \in V_0} \int_\Omega |\grad v|^2\,dx .
    \end{equation*}
    Then, one simply infers the spaces of each function from the definition of the Gateaux derivative. 
    \item The solution $u$ was formulated in a space without boundary condition. This is important because the regularity theory will depend on the solution space being a Hilbert space, and the space $V_g$ is not even a vector space as it is not closed under addition. This can be solved by defining adequate \emph{lifting} operators, i.e. a function $G$ in $H^1(\Omega)$ such that $\gamma_D G = g$ that allows us to write $u$ in $V_g$ as 
    \begin{equation*}
        u = u_0 + G,
    \end{equation*}
    where $u_0$ belongs to $V_0$. We can then rewrite the problem in $V_g$ as a problem in $V_0$ (and I encourage the reader to do this procedure at least once in their life). The existence of a lifting function in this case is given by the surjectivity of the Dirichlet trace, but it can be tricky in other contexts. This is also tricky in nonlinear problems, which justifies that nonlinear problems are typically studied with homogeneous boundary conditions.  
    \item We note that the Laplacian is now being interpreted as a \emph{distribution}, and thus the strong problem (including the boundary conditions) yields the definition of the \emph{action} of the distribution. In particular, this means that the action of the distribution naturally changes with the boundary conditions. This observation is fundamental to understand Discontinuous Galerkin methods, or other formulations defined on broken spaces (i.e. spaces that allow for discontinuities). 
\end{itemize}

To conclude this section, we wanted to establish that if the solution is sufficiently regular, then the weak form is equivalent to the strong form. The main tool for this is the fundamental lemma of the calculus of variations: 

\begin{lemma}[Fundamental lemma of the calculus of variations]\label{lemma:lema-calvar}
    Consider $\Omega\subset \R^d$ bounded and set $f\in L^1(\Omega)$ such that
    \begin{equation}\label{eq:lemma-calvar}
        \int_\Omega f\varphi\,dx = 0 \qquad \forall \varphi \in C_0^\infty(\Omega).
    \end{equation}
    Then, $f=0$ almost everywhere.
\end{lemma}

Consider now the weak formulation of the Poisson problem: Find $u$ in $V_0$ such that 
\begin{equation}
    (\grad u, \grad v) = (f, v) - (\grad u_g, \grad v) + \langle t, v\rangle_{-1/2, 1/2},
\end{equation}
where $f$ and $\vec t$ are integrable. Then, integrating by parts one obtains
\begin{equation}\label{eq:ibp-nonhomogeneous-poisson}
    (-\Delta (u+u_g) - f, v)+\langle \grad u \cdot \vec n, v\rangle = \langle t, v\rangle_{-1/2,1/2}.
\end{equation}
If we consider test functions $v$ in $C_0^\infty(\Omega)$, then the problem reduces to
\begin{equation}
    (-\Delta (u+u_g) - f, v) = 0,
\end{equation}
and if $u$ and the lifting function $u_g$ are integrable, then Lemma~\ref{lemma:lema-calvar} yields
\begin{equation*}
    -\Delta \tilde u = f,
\end{equation*}
which is the strong form for the combined solution $\tilde u= u + u_g$. Substituting this back into~\ref{eq:ibp-nonhomogeneous-poisson}, we obtain the weak form
\begin{equation*}
    \langle \grad u \cdot \vec n, v\rangle = \langle t, v\rangle.
\end{equation*}
Using again Lemma~\ref{lemma:lema-calvar} on the subspace topology of $C_0^\infty(\Gamma_N)$, then we obtain the equation
\begin{equation*}
    \grad u\cdot \vec n = t,
\end{equation*}
which holds strongly.\\

\example{
We encourage the reader to try to compute the weak formulation of the Poisson problem in mixed form. To do this, one must define the auxiliary variable $\vec \sigma \coloneqq \grad u$, so that the strong form of the problem now becomes
\begin{equation*}
    \begin{aligned}
        -\dive \vec\sigma &= f &&\text{ in $\Omega$}, \\
        \vec \sigma - \grad u &= 0 &&\text{ in $\Omega$}, \\
        \gamma_D u &= g &&\text{ on $\Gamma_D$}, \\
        \gamma_N \vec\sigma &= h &&\text{ on $\Gamma_N$}. 
    \end{aligned}
\end{equation*}
This problem will be studied in detail further ahead. 
}

\section{Elliptic problems and the Lax-Milgram lemma}
One of the easiest classes of PDEs where we can prove existence and uniqueness of solutions are \textit{elliptic} problems, whose weak forms have elliptic terms.
\begin{definition}[Elliptic forms]\label{def:elliptic-form}
A bilinear form $a(\cdot, \cdot)$ defined on a Hilbert space $X$ is said to be elliptic if there exists a constant $\alpha$ such that
\begin{equation}\label{eq:elliptic-form}
    a(x, x) \geq \alpha \| x \|^2_X \qquad \forall x\in X.
\end{equation}
\end{definition}
This property is the basis of the Lax-Milgram lemma, which gives sufficient conditions for the existence and uniqueness of solutions. 
\begin{lemma}[Lax-Milgram]\label{lemma:lax-milgram}
    Consider a bounded bilinear form $a: H\times H\to \R$ defined on a Hilbert space $H$ that is elliptic with constants $C$ and $\alpha$ respectively, and a linear functional $f$ in $H'$. Then, there exists a unique $u$ in $H$ such that 
    \begin{equation}\label{eq:lax-milgram-form}
        a(u, v) = f(v) \qquad \forall v \in H.
    \end{equation}
    This solution is continuous with respect to the data, in the sense that there exists a positive constant $C$ such that 
    \begin{equation}\label{eq:lax-milgram-a-priori}
        \| u\|_H \leq \frac 1 \alpha \| f \|_{H'} .
    \end{equation}
    This is typically referred to as the \emph{a priori} estimate. 
\end{lemma}

Before providing a proof, we note that every continuous bilinear form $a:H\times H\to \R$ induces an operator $A:H\to H'$ given by
\begin{equation}\label{eq:form-induced-by-matrix}
    (Au)[v] = a(u,v),
\end{equation}
which one could also write as $Au = a(u, \cdot)$. Naturally, the bilinear form $a$ is bounded if and only if the operator $A$ is bounded. 

\begin{proof}
    It will be seen further ahead that this can be easily proved using the inf-sup condition. Still, we present a more elementary proof that uses only the properties of the bilinear form and a fixed point argument. Consider $\rho>0$ and the fixed-point map $T:H\to H$ given by 
    \begin{equation*}
        T(u) = u - \rho \mathcal R^{-1}\circ (Au - F),
    \end{equation*}
    where it can be seen that $T$ is linear, and $\mathcal R$ is the Riesz map between $H$ and $H'$. Now, we look for $\rho$ such that $T$ is a contraction, which we do simply by hand. Consider thus two functions $u,V$ in $H$, then: 
    \begin{tightalign*}
        \| T(u) - T(v)\|_H^2 &= \|T(u - v) \|_H^2 \\
        &= (u-v, u-v)_H - 2\rho(u-v, \mathcal R^{-1}\circ A(u-v))_H\\
        &\phantom{=  } + \rho^2(\mathcal R^{-1}\circ A(u-v), \mathcal R^{-1}\circ A(u-v))_H \\
        &= \|u-v\|_H^2 - 2\rho\langle A(u-v), u-v\rangle_{H'\times H} + \rho^2 \| \mathcal R^{-1} \circ A(u-v)\|_H^2.
    \end{tightalign*}
    We bound the second and third terms as follows: 
    \begin{tightalign*}
        \langle A(u-v), u-v\rangle &= a(u-v, u-v)  \tag{by definition of $A$}\\
        &\geq \alpha \| u-v\|_H^2, \tag{ellipticity}
    \end{tightalign*}
    \begin{tightalign*}
        \| \mathcal R^{-1} \circ A(u-v) \|_H &= \| A(u-v) \|_{H'} \tag{Riesz isometry}\\
        &= C \| u-v \|_H^2. \tag{continuity}
    \end{tightalign*}
    Plugging this into our previous estimate we get
    \begin{equation*}
        \| Tu - Tv \|_H \leq (1 - 2\rho \alpha + \rho^2 C^2)^{1/2}\| u-v \|_H ,
    \end{equation*}
    which shows that $T$ is a contraction whenever $\rho\in (0,\frac{2\alpha}{C^2})$. Stability follows naturally from the properties of $a$:
    \begin{tightalign*}
        \alpha \| u \|^2 & \leq a(u,u) \tag{ellipticity}\\
        & = F(u) \\
        & \leq \| F \|_{H'} \|u \|_H,
    \end{tightalign*}
    which shows the desired stability estimate: 
    \begin{equation*}
        \| u \|_H \leq \frac{1}{\alpha} \| F \|_{H'}.
    \end{equation*}
\end{proof}

\section{Examples}
\paragraph{The Poisson problem} Consider $f$ in $H^{-1}(\Omega)$ and $g$ in $H^{1/2}(\Gamma)$ with $\Gamma\coloneqq \partial\Omega$. The Poisson problem in strong form is given as the following boundary value problem: 
\begin{equation*}
    \begin{aligned}
        -\Delta u  &= f &&\tin\Omega\\
        \gamma_0 u &= g &&\ton\Gamma.
    \end{aligned}
\end{equation*}
Note that the strong form must be understood in the distributional sense, i.e. as an equation in $H^{-1}(\Omega)$. To derive the weak formulation, consider a function $v$ in $H_0^1(\Omega)$, then using the boundary conditions we obtain that 
\begin{equation}
    -\langle \Delta u,v\rangle = (\grad u, \grad v),
\end{equation}
where $(\cdot, \cdot)$ is the $L^2(\Omega)$ product. Thus the weak formulation reads: Find $u$ in $H_0^1(\Omega)$ such that 
    \begin{equation}
\int_\Omega \grad u\cdot \grad v\,dx = \langle f, v\rangle \qquad \forall v\in H_0^1(\Omega).
\end{equation}
This problem can be shown to be well-posed using Lax-Milgram's lemma and the Poincaré inequality. Small exercise: extend the proof to the case of non-homogeneous Dirichlet boundary conditions.

In the case of having a boundary condition defined only on a portion $\Gamma_D$ of the boundary, the formulation changes, because (i) we need further information regarding the Neumann trace on the complement of the boundary, (ii) the test space looks different. In particular, we define the solution space given by 
\begin{equation}\label{eq:def-V0}
    V_0 = \{v\in H^1(\Omega): \quad v = 0 \quad\text{ on $\Gamma_D$}\},
\end{equation}
which using the generalized Poincaré can be shown to still satisfy an ellipticity estimate. 

\paragraph{The pure Neumann problem} In general, having Neumann boundary conditions is problematic for two reasons: (i) it results in a \emph{data compatibility} condition, and (ii) it results in having a non-trivial kernel in the problem. The problem in general reads: find $u$ in $H^1(\Omega)$ such that
\begin{equation}
    \begin{aligned}
        -\Delta u &= f && \text{in $\Omega$},\\
        \grad u \cdot \vec n &= h && \text{on $\partial\Omega$}.
    \end{aligned}
\end{equation}
The weak formulation is 
\begin{equation}
    (\grad u, \grad v) = \langle f, v \rangle \qquad \forall v\in H^1(\Omega),
\end{equation}
where it is easy to see that if $u$ is a solution, then $u+c$ is also a solution for all $c\in \R$. This means that the problem has a kernel, which is given by the space of constant functions, i.e. $\texttt{span}(\{1\})$. The other problem is that, when one considers a test function in the kernel of the problem, this yields the following: 
\begin{equation}
    (\grad u, \grad 1) = 0 = \langle f, 1\rangle.
\end{equation}
This is a compatibility condition on the data, and it shows that having compatible data is \emph{necessary} for having a well-posed formulation. Because of these reasons, one considers a solution (and test) space that is orthogonal to the kernel: 
\begin{equation}
    V = \{u\in H^1(\Omega): \int_\Omega u \,dx = 0\},
\end{equation}
where the null average condition can be seen as 
\begin{equation}
    \int_\Omega u \,dx = (u, 1)_0 = (u,1)_0 + (\grad u, \grad 1)_0 = (u, 1)_1,
\end{equation}
and thus the orthogonality is being considered with respect to the natural space $H^1(\Omega)$. With it, the weak formulation is given as: Consider $f$ a compatible function in $H^{-1}(\Omega)$, then find $u$ in $V$ such that
\begin{equation}
    (\grad u, \grad v) = \langle f, v\rangle \qquad \forall v\in V.
\end{equation}