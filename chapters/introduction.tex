\section*{Introduction}

These notes started as backup material for a course on some deeper topics in mathematical engineering at Pontificia Universidad Católica de Chile, the second semester of 2024, but then evolved to include more introductory material on Numerical Analysis of PDEs and Continuum Mechanics for our course \emph{Applications of Functional Analysis and PDEs in Engineering}. The hand-written notes by Federico Fuentes were absolutely fundamental to provide depth and context to most of the content presented. The final cornerstone of these notes was the typesetting effort done by Bastián Herrera, whose knowledge on these topics has provided a constant input for perfectioning the contents and the coherence of the presentation. We also thank the School of Engineering at Pontificia Universidad Católica de Chile for the funding. In general, these notes put together contents found online and in books more than producing new material of proofs of theorems. Beyond the books being cited, which are in general fantastic references, we tried to acknowledge many of the available notes online that provide fantastic level of detail and insight.

The idea of these notes is to provide mathematical tools for students that give them the ability to assess the difficulty of mathematical problems, mainly within the world of partial differential equations (PDEs). The target is ultimately to implement these models, so that all tools are oriented towards having solid foundations that allow one to trust a computational model. Informally speaking, the main mathematical concepts to haunt us throughout all these notes are: 
\begin{itemize}
    \item Existence and uniqueness: it is a natural baseline in the mathematician's world to try to solve only problems that \emph{have} a solution. Otherwise, things might be as pointless as developing an iterative method for finding real numbers such that $x^2 = -1$. Uniqueness is a further luxury, but sometimes two different methods give two different solutions, and having only those things at hand can make it difficult to distinguish whether that is a bug or a feature of the model. There exist some root-isolation methods that allow to find solutions of a problem that are \emph{different} from a given one. This is out of the scope of this course. 
    \item Stability: the intuitive idea behind this is that small perturbations in the data give rise to small changes in the solution. This typically looks like 
    \begin{equation*}
        \| u\|_X \leq C\| f\|_{X'},
    \end{equation*}
    where $u$ is the solution of a problem that depends on $f$, and $X$ is some functional (hopefully Hilbert) space with dual $X'$. More rigorously, this means that the solution map $f \mapsto u(f)$ is bounded, or continuous in the linear case. Stability also sometimes refers to time dynamics and the fact that a discrete solution stays \emph{within a certain distance} of the true solution throughout a simulation. In the continuous setting, it might also mean that there are no finite-time singularities. In general, stability is not a well defined term, but still a widely understood one to anyone who has struggled to get a code to run correctly, and a highly desired property. 
\end{itemize}
All other properties (or at least most of them anyway) are ways to guarantee that a problem enjoys one of these nice properties. There are ways to handle problems that do not have those properties, but they are almost always extremely problem-dependent, and the person studying such problems should dive deep into the sectorial knowledge to see how certain communities deal with such issues. This is an aspect that mathematically-oriented people almost always disregard, which has some severe mathematical (and social) consequences. In fact, some extremely classical models in engineering are still far from understood mathematically, such as the Navier-Stokes equations. This has not prevented the computational fluid dynamics (CFD) community from solving these models with extreme efficiency, and from further leveraging them for industrial applications which, unsurprisingly, work fantastically. Discovering the amazing ways in which mathematically agnostic communities solve mathematically hard problems is, and will probably be for very long, a beautiful opportunity for collaboration.


