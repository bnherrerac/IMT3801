\section*{Introduction}

These notes exist as backup material for a course on some deeper topics in Math Eng at Pontificia Universidad Católica de Chile, the 2nd semester of 2024. The idea is to provide mathematical tools for students that give them the ability to assess the difficulty of mathematical problems, mainly within the world of Partial Differential Equations (PDEs). The target is ultimately to implement these models, so that all tools are oriented towards having solid foundations that allows one to trust a computational model. Informally speaking, the main mathematical concepts to haunt us during all these notes are: 
    \begin{itemize}
        \item Existence and uniqueness: It is a natural baseline in the mathematician's world to try to solve only problems that \emph{have} a solution. Otherwise, things might be as pointless as developing an iterative method for finding real numbers such that $x^2 = -1$. Uniqueness is a further luxury, but sometimes two different methods give two different solutions, and having only those things at hand can make it difficult to distinguish whether that is a bug or a feature of the model. There exist some root-isolation methods that allow to find solutions of a problem that are \emph{different} from a given one. This is out of the scope of this course. 
        \item Stability: The intuitive idea behind this is that small perturbations in the data give rise to small changes in the solution. This typically looks like 
            $$ \| u\|_X \leq \| f\|_{X'}, $$
        where $u$ is the solution of a problem that depends on $f$, and $X$ is some functional (hopefully Hilbert) space. More rigorously, this means that the solution map $f \mapsto u(f)$ is bounded, or continuous in the linear case. Stability also sometimes refers to time dynamics and the fact that a discrete solution stays \emph{within a certain distance} of the real solution throughout a simulation. In the continuous setting, it might also mean that there are no finite-time singularities. In general, stability is not a well defined term, but still a widely understood one to anyone who has struggled to get a code to run correctly, and a highly desired property. 
    \end{itemize}
All other properties (or at least most of them anyway) are ways to guarantee that a problem enjoys one of these nice properties. There are ways to handle problems that do not have those properties, but they are almost always extremely problem dependent, and the person studying such problems should dive deep into the sectorial knowledge to see how certain communities deal with such issues. This is an aspect that mathematically oriented people almost always disregard, which has some severe mathematical (and social) consequences. In fact, some extremely classical models in engineering are still far from understood mathematically, such as the Navier-Stokes equations. This has not prevented the CFD community from solving these models with extreme efficiency, and from further leveraging them for industrial applications which, unsurprisingly, work fantastically. Discovering the amazing ways in which mathematically obvlivious communities solve mathematically hard problems is, and will probably be for very long, a beautiful opportunity for collaboration.


We strongly thank Bastián Herrera for all his help in converting floating drafts and hand-written notes into a full-fledged set of coherent notes. We also thank the School of Engineering at Pontificia Universidad Católica de Chile for providing resources for such important teaching tasks.
