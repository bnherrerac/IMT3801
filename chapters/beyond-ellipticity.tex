In the previous section we have thorougly studied elliptic problems and many approximation propoerties. Still, one might rightfully notice that ellipticity can be quickly broken. For example, the operator $-\Delta: H_0^1(\Omega)\to H^{-1}(\Omega)$ is elliptic, but if we remove the minus sign, it loses that property. As this case is linear, it is possible to remap the unknown with $u\mapsto -u$, but it still feels unsatisfactory that the property is so fragile. Because of this, in this section we review two important theories that go beyond ellipticity: the inf-sup theory and the theory of Fredholm operators.
%%%%%%%%%%%%%%%%%%%%%%%%%%%%%%%%%%%%%%%%%%%%%%%%%%%%
\section{Inf-sup conditions}

We are interested in showing first that the inf-sup condition means 
surjectivity. This will require the use of certain results from 
functional analysis. For detailed proofs and omitted details, see \cite{chen2024infSup} 
and \cite{gatica2014simple}. A first important result is to characterize 
operators with closed range. 

\begin{lemma}
    Let \(U\) and \( V\) be Banach spaces and \(T:U\to V\) a linear 
    continuous operator. Then, \(T\) is injective and its range \(R(T)\) is 
    closed if and only if  \(T\) is bounded below, i.e., there exists 
    a positive constant \(c\) such that 
    \begin{displaymath}
        \lVert Tu \rVert \ge c\lVert u \rVert
        \quad \text{ for all } u\in U
    \end{displaymath}
\end{lemma}
\begin{proof}
    First, we assume \(T\) is bounded below. If \(Tu=0\), the inequality 
    implies \(u=0\), i.e., \(T\) is injective. Let \(\{Tu_k\}\) be a convergent sequence in \(V\). By the inequality, we have that
    \begin{displaymath}
        \lVert Tu_k - Tu_m \rVert \ge c\lVert u_k - u_m \rVert
    \end{displaymath}
    for some \(k,m\). Then, because \(\{Tu_k\}\) is a Cauchy sequence, we know that \(\{u_k\}\) is also a Cauchy sequence and, because both spaces are Banach, the sequence then converges to some \(u\in U\). The continuity of \(T\) shows that \(Tu_k\) converges to \(Tu\) and thus the range of \(T\) is closed. 

    Now, assume \(T\) is injective and its range is closed. Then, because 
    \(R(T)\) is a closed subspace of a Banach space, it is also Banach. As \( T\) is injective, \(T^{-1}\) is well defined on \(R(T)\), i.e., 
    \(T:U\to R(T)\subset V\) is invertible. In addition, Open Mapping Theorem 
    shows that \(T^{-1}\) is continuous. Then
    \begin{displaymath}
        \lVert u \rVert = \lVert T^{-1}(Tu) \rVert 
        \le \lVert T^{-1} \rVert \lVert Tu \rVert 
    \end{displaymath}
    which implies \(T\) is bounded below with constant 
    \(c = \lVert T^{-1} \rVert^{-1}\).
\end{proof}

Now, we would like to characterize the surjectivity of the operator using its dual.

\begin{lemma}
    Let \(U\) and \(V\) be Banach spaces and let \(T\) be a linear continuous 
    operator. \(T\) is surjective if and only if \(T'\) is injective with closed 
    range, where \(T'\) is the dual (or transpose) operator of \(T\).
\end{lemma}
\begin{proof}
    First, if \(T\) is surjective, then \(R(T) = V\) is closed. By Closed Range 
    Theorem, \(R(T')\) is also closed and 
    \begin{displaymath}
        R(T) = N(T')^\bot = 
        \{v\in V: \langle v', v\rangle = 0 \,\,\, \forall v' \in N(T')\}
    \end{displaymath}
    where \(N(T')\) is the null space of \(T'\). Now, since 
    \(V = R(T) = N(T')^\bot\), then, for each \(v' \in N(T')\), it holds
    that \( \langle v', v\rangle = 0\,\,\, \forall v \in V\), which means that 
    \(v' = 0\). Thus, \(N(T') = \{0\}\), so \(T'\) is injective.

    For the proof in the other direction, we get from Closed Range Theorem that also $R(T)$ is closed. By contradiction, consider $v$in $V$ such that $v\not\in R(T)$. By Hahn-Banach we get that there exists $f$ in $V'$ such that $f(R(T)) = 0$ and $f(v)=1$\footnote{This is an improvement over Urysohn's Lemma, which yields only a possibly nonlinear function}. Then, $T'f$ is such that
        \begin{equation*}
\langle T'f', u\rangle = \langle f, Tu\rangle = u \qquad \forall u\in U.
\end{equation*}
    This implies that $f=0$, which contradicts that $f(v)=1$. 

\end{proof}

Combining both lemmas, we can write
\begin{align*}
    T \text{ surjective}
    &\Longleftrightarrow T'\text{ injective and with closed range}
    \\&\Longleftrightarrow T'\text{ is bounded below}
\end{align*}

Let \(U\) and \(V\) Hilbert spaces and \(T:U\to V\) linear continuous operator
with transpose \(T':V'\to U'\), considering its adjoint operator \(T^*:V \to U\), 
we can write \[T^* = R_U^{-1} \circ T' \circ R_V\] and 
\[\lVert T^* \rVert_{V\to U} = \lVert T' \rVert_{V'\to U'}\] 
where \(R_U: U\to U'\) and 
\(R_V: V\to V'\) are Riesz operators. Then, using the Riesz representation 
Theorem, we have that

\begin{align*}
    T \text{ surjective}
    &\Longleftrightarrow 
    T'\text{ is bounded below}
    \\&\Longleftrightarrow 
    \lVert T'v' \rVert_{U'} \ge c \lVert v' \rVert_{V'} \quad \forall v'\in V'
    \\&\Longleftrightarrow 
    \lVert (T'\circ R_V) (v) \rVert_{U'} \ge c \lVert v \rVert_{V} 
    \quad \forall v\in V
    \\&\Longleftrightarrow 
    \lVert (R_U \circ R_U^{-1} \circ T'\circ R_V) (v) \rVert_{U'} 
    \ge c \lVert v \rVert_{V} \quad \forall v\in V
    \\&\Longleftrightarrow 
    \lVert (R_U \circ T^*) (v) \rVert_{U'} 
    \ge c \lVert v \rVert_{V} \quad \forall v\in V
    \\&\Longleftrightarrow 
    \lVert T^* v \rVert_{U} 
    \ge c \lVert v \rVert_{V} \quad \forall v\in V
    \\&\Longleftrightarrow 
    c \lVert v \rVert_{V}  
    \le \sup_{u\in U}\frac{(Tu, v)}{\lVert u \rVert_{U}} \quad \forall v\in V
    \\&\Longleftrightarrow 
    0 < c \le \inf_{v\in V}\sup_{u\in U}
    \frac{(Tu, v)}{\lVert u \rVert_{U}\lVert v \rVert_{V}}
\end{align*}

We also can characterize the injectivity of the operator.

\begin{lemma}
    Let \(U\) and \(V\) be Banach spaces and let \(T:U\to V\) be a linear 
    continuous operator. Then, \(T\) is injective if and only if 
    \[\sup_{v'\in V'} \langle Tu, v'\rangle_{V\times V'} > 0 
    \quad \forall u\in U\text{, }u \neq 0\]
\end{lemma}
\begin{proof}
    Suppose that \(T\) is injective. If \(u\neq 0\), then \(Tu \neq 0\), and so, 
    for some \(v'\in V'\), we have that \(\langle Tu, v'\rangle_{V\times V'} 
    \neq 0\). Which implies the right side of the equivalence.

    Now, assume the right side of the equivalence. By contradiction, if \(T\)
    is non injective, there exists \(u\in U \) with \(u\neq 0\) such that \(Tu = 0\), 
    which contradicts the hypothesis.
\end{proof}

We can now postulate a general Lax-Milgram Theorem. We write it in Hilbert Spaces.

\begin{theorem}(Generalized Lax-Milgram)
    Consider \(H_1\), \(H_2\) Hilbert spaces and a bounded bilinear form 
    \(B:H_1\times H_2 \to \mathbb{R}\). Then, there exists a unique \(u\in H_1\) for each 
    \(F\in H_2'\) such that \[B(u,v) = F(v),\quad \forall v \in H_2\] if and only if 
    \begin{enumerate}
        \item \(\exists \alpha > 0\) such that \[\sup_{u\in H_1,u\neq 0}
        \frac{B(u,v)}{\lVert u \rVert_{H_1}} \ge \alpha \lVert v \rVert_{H_2}
        \quad \forall v\in H_2\quad \text{ (surjective) }\]
        \item \[\sup_{v\in H_2} B(u,v) > 0 
        \quad \forall u\in H_1,\,\, u\neq 0
        \quad \text{ (injective) }\]
    \end{enumerate}
\end{theorem}
We note that injectivity can be equivalently stated as an inf-sup condition simply by inverting the arguments in the surjectivity inf-sup.  The main difficulty in using this theory is that well-posedness of a discrete problem is not inherited from the continuous one. In fact, consider $U_N\subset U$ and $v'_N\in V'_N\subset V$ discrete spaces, then 
\begin{equation*}
\|T'v_N'\|_{U'} = \sup_{u\in U}\frac{\langle u, T'v_N'\rangle}{\|u\|_U} \geq \sup_{u_N\in U_N}\frac{\langle u_N, T'v_N'\rangle}{\|u_N\|_U},
\end{equation*}
which establishes our claim. Despite this, one can still recover convergence as before. To see this, consider $u$ and $u_N$ the continuous and discrete solutions given by the following problems with $a:U\times V \to \R$:
    \begin{equation*}
\begin{aligned}
        a(u,v) &= f(v) &&\forall v\in V,\\
        a(u_N, v_N) &= f(v_N) &&\forall v_N\in V_N.
    \end{aligned}
\end{equation*}
This bilinear forms make sense as operator equations as they induce an $A:U\to V'$. Both problems are well-posed are both operators are surjective, which implies that 
    \begin{equation*}
a(u-u_N, v) = 0  \qquad \forall v_N \in V_N.
\end{equation*}
From the injectivity seen as an inf-sup condition we get that
\begin{equation*}
\alpha \| u - u_N \|_U \leq \sup_{v_N\in V_N} \frac{a(u-u_N, v_N)}{\|v_N\|} \leq \| a \| \|u - \xi_N \|,
\end{equation*}
for some $\xi_N$ in $U_N$, as in the Lax-Milgram proof. Taking the infimum in $\xi_N$ and setting $C=\|a\|$ yields that
    \begin{equation*}
\| u - u_N \| \leq \frac{C}{\alpha}\text{dist}(u, U_N).
\end{equation*}


%%%%%%%%%%%%%%%%%%%%%%%%%%%%%%%%%%%%%%%%%%%%%%%%%%%%
\section{Fredholm operators}

Most of the forms in which we present our problems comes from the fantastic notes from Andrea Moiola on time-harmonic acoustic waves \cite{moiola2021scattering}. Our reference problem will be the Helmholtz equation, given by the following strong form:

\[
\left\{
\begin{array}{rc}
    -\Delta u -k^2 u = f, &\Omega,\\
    u=0, &\partial\Omega,
\end{array}\right.
\]

for some $f\in H^{-1}(\Omega)$ and some boundary condition. This equation can be recoverred in the following way: consider the wave equation
    \begin{equation*}
\ddot u - \Delta u = 0,
\end{equation*}
and consider a variable separation procedure with $u(t,x) = V(t)U(x)$. This is a common technique, and it results in 
    \begin{equation*}
\frac{\Delta U}{U} = k^2.
\end{equation*}
multiplying by $U$ gives the desired result. Regarding well-posedness, let's first explore what we can conclude with Lax-Milgram. We will assume $u$ in $H_0^1(\Omega)$ for simplicity, which yields the following weak form: Find $u$ in $H_0^1(\Omega)$ such that

\[
a(u,v) = (\nabla u, \nabla v) - k^2(u,v) = \langle f, v\rangle, \quad \forall v\in H_0^1.
\]

Using $H_0^1(\Omega)$ with the norm induced by the seminorm $|v|_1 = \|\nabla v\|_0$, Poincaré inequality gives us $\|u\|_0\leq C_{\Omega}\|\nabla u\|_0$, and so the Lax-Milgram hypotheses look as follows: 

\begin{itemize}
    \item Boundedness: 
    \begin{align*}
        a(u,v) &\leq \|\nabla u\|_0\|\nabla v\|_0 + k^2\|u\|_0\|v\|_0,\\ 
        &\leq (1+k^2C_\Omega^2)|u|_1|v|_1.
    \end{align*}

    \item Coercivity: 
    \begin{equation*}
        a(v,v) = \|\nabla v\|^2_{0,\Omega} - k^2\|v\|^2_{0,\Omega}.
    \end{equation*}

    We note that, by Poincaré inequality,

    \[ k^2\|v\|^2_0 \leq k^2 C^2_{\Omega}\|\nabla v\|^2_0, \]

    and thus,

    \begin{align*}
        a(v,v) &\geq \|\nabla v\|^2_0 - k^2C_\Omega^2\|\nabla v\|^2_0 \\
        &= (1- k^2 C_\Omega^2)\|\nabla v\|^2_{0,\Omega}.
    \end{align*}

    In other words, the problem is well-possed if 
    \begin{equation*}
1-k^2C_\Omega^2 > 0 \iff k^2 < \frac{1}{C_\Omega^2}
\end{equation*}
\end{itemize}

This is a very limited answer, so we will now answer what happens for arbitrary $k\in\R$.

\begin{definition}
    A linear operator $K:H_1 \to H_1$ is \textbf{compact} if the image of a bounded sequence admits a converging subsequence.
\end{definition}
\begin{definition}
    A bounded linear operator is a \textbf{Fredholm operator} if it is the sum of an invertible and a compact operator. 
\end{definition}

\begin{theorem}[Fredholm alternative]
    A Fredholm operator is injective if and only if it is surjective. In such case, it has a bounded inverse.
\end{theorem}

A simpler way of showing that an operator is Fredholm, is through a Gårding inequality.

\begin{definition}
    Consider $H\subset V$ Hilbert spaces with a continuous embedding $H\hookrightarrow V$. 
    A bilinear form $a:H\times H\to \R$ satisfies a \textbf{Gårding inequality} if there exist two positive constants $\alpha$, $C_V$ subject to, 
    
    \[a(v,v) \geq \alpha \|v\|^2_H - C_V\|v\|^2_V,\quad \forall v\in H.\]
\end{definition}

\textbf{Proposition:}
 If the inclusion $H\hookrightarrow V$ is compact, then the operator $A: H\to H^*$ associated to $a:H\times H\to \R$, i.e,
 \[\langle Ax,y\rangle = a(x,y),\]

 is such that if $a$ satisfies Gårding then $A$ is Fredholm. 

\begin{proof}
    We note that, by definition of Gårding,  \(a(u,v) + C_V(u,v)_V\) is invertible because of Lax-Milgram. 
    We now try to write the equation in operator form. The form $(u,v)_V$ is handled as follows: set $T:V\to H^*$ as

    \[\langle Tv, w\rangle_{H^*\times H} = "(v,w)_V"\equiv (v,iw)_V,\]
    where $i: H\to V$ is the compact embedding. $T$ is clearly bounded:

    \begin{align*}
        \|Tv\|_{H^*} &= \sup_{w\in H}\frac{(v,iw)_V}{\|w\|_H},\\
        &\leq \sup_{w\in H}\frac{\|v\|_V\|iw\|_V}{\|w\|_H},\\
        &\leq \sup_{w\in H}\frac{\|v\|_V\|i\|\|w\|_H}{\|w\|_H},\\
        &\leq \|i\|\|v\|.
    \end{align*}

    Then, the operator associated to the problem is $B :=A + C_V T\circ i$, where:
    \begin{itemize}
        \item $B$ is invertible,
        \item $T$ is continuous,
        \item $i$ is compact. 
    \end{itemize}

    The last two impliy that $T\circ i$ is compact, as the composition of continuous and compact operators is compact. 
    It follows that $A = B - C_V T\circ i$ is Fredholm.
\end{proof}



\section{Galerkin stability}

We have seen that elliptic problems have the following a-priori stability estimate

\[\alpha \|u\|_H \leq \|f\|_{H^*}\quad\text{ for }\quad a(u,v) = \langle f, v\rangle, \quad \forall v\in H.\]

which carries naturally to the discrete problem:

\[\alpha \|u_h\|_H \leq \|f\|_{H^*}\quad\text{ for }\quad a(u_h,v_h) = \langle f, v\rangle, \quad \forall v_h\in H_h.\]

This can be rewritten as follows: denote the orthogonal projection $\Pi_h: H\to H^*$, and $R_H:H\to H^*$ the Riesz map.

Then:

\begin{align*}
    &a(u,v) = \langle f, v\rangle_{H^*\times H}\quad \forall v\in H,\\
    \iff &\langle Au, v\rangle_{H'\times H} = \langle f, v\rangle_{H'\times H}\quad \forall v\in H,\\
    \iff &Au = f \quad\text{ in $H'$ and }(R^{-1}\circ A)u = R^{-1}f\quad\text{in }H.
\end{align*}

Also:

\begin{align*}
    &a(u_h,v_h) = \langle f, v_h\rangle\quad\forall v_h\in H_h,\\
    \iff &(R^{-1}\circ A u_h,v_h)_H = (R^{-1}f,v_h)_H\quad\forall v_h\in H_h,\\
    \iff &\Pi_h\mathbb{A}u_h = \Pi_h\mathbb{F},
\end{align*}

where $\mathbb{A} = R^-1\circ A$ and $\mathbb{F} = R^{-1}f$. The stability estimate gives, 

\[\|u_h\| \leq\frac{1}{\alpha}\|\Pi_h\mathbb{F}\| = \frac{1}{\alpha}\|\Pi_h\mathbb{A} u_h\|.\]

Finally, by duality, we get

\begin{align*}
    \|u_h\| &\leq \frac{1}{\alpha}\|\Pi_h\mathbb{A}u_h\| = \frac{1}{\alpha}\sup_{v_h}\frac{(\Pi_h\mathbb{A}u_h,v_h)}{\|v_h\|}
    =\frac{1}{\alpha}\sup_{v_h}\frac{\langle Au_h,v_h\rangle_{H'\times H}}{\|v_h\|} = \frac{1}{\alpha}\sup_{v_h}\frac{a(u_h,v_h)}{\|v_h\|}.
\end{align*}

We call this result,

\[\|u_h\| \leq C\sup_{v_h\in V_h}\frac{a(u_h,v_h)}{\|v_h\|},\qquad \forall u_h\in H_h,\]

\todo[inline]{NB: Esto ya no tiene sentido acá.}
an \textbf{inf-sup condition}, which we will later use to prove \textbf{surjectivity} and \textbf{injectivity} separately. One may readily see that this condition implies (discrete) injectivity as $Au_h$ implies $u_h = 0$.

We will require the following Lemma regarding discrete stability of Fredholm operators. Details are provided in \cite{sayas2019variational}.
\begin{lemma}
Consider a bilinear form $a:H\times H\to \R$ associated to an injective Fredholm operator $A+K$. Then, there exists $C, h_0>0$ such that

\[\|u_h\|_H\leq C\sup_{v_h}\frac{a(u_h,v_h)}{\|v_h\|},\quad \forall u_h\in H_h, \quad h\leq h_0.\]
In other words, discrete stability holds only for sufficiently fine meshes. 
\end{lemma}

Consider the discrete problem
\[a(u_h,v_h) + b(u_h,v_h) = \langle f, v_h\rangle\]

where $a$ and $b$ are the bilinear forms associated to an elliptic and a compact operator respectively, and consider the Galerkin projection $G_h: H\to H_h$

\[a(G_hu, v_h) + b(G_hu, v_h) = a(u, v_h) + b(u,v_h).\]

Under the previous hypothesis, for $h\leq h_0$ we have, 

\[\|G_hu\|\leq C\sup_{v_h}\frac{a(G_hu, v_h) + b(G_hu, v_h)}{\|v_h\|}\leq C\|A+K\|\|u\|.\]

Then $G_h$ is bounded. We observe that, as $G_h$ is a porjection, it holds that $G_h\Pi_h = \Pi_h$, and thus:
\begin{align*}
    \|u - G_hu\|&\leq \|u-\Pi_hu\| + \|\Pi_hu - G_hu\|,\\
    &= \|u-\Pi_hu\| + \|G_h(\Pi_hu - u)\|,\\
    &\leq (1+C\|A+K\|)\|u - \Pi_h u\|,
\end{align*}

which implies

\[\|u-u_h\|\leq (1+C\|A+K\|)\inf_{v_h\in H_h}\|u-v_h\|,\] 

which is a Céa estimate for sufficiently small $h$.

%%%%%%%%%%%%%%%%%%%%%%%%%%%%%%%%%%%%%%%%%%%%%%%%%%%%
\section{Saddle point problems}

In this section we will study the well-posedness theory of saddle point problems. The presentation has been taken, mostly verbatim, from \cite{gatica2014simple}. A saddle point problem has the form:
\begin{displaymath}
    \begin{bmatrix}
        A & B^{T}\\ 
        B & 0
    \end{bmatrix}
    \begin{bmatrix}
        u \\ p
    \end{bmatrix}
    =
    \begin{bmatrix}
        f \\ g
    \end{bmatrix}.
\end{displaymath}

To grasp the relevance of this formulations, let's first see
some examples:
\begin{itemize}
    \item \textbf{Stokes}: The variational formulation of
    Stokes reads:
    \begin{displaymath}
    \begin{cases}
        (\nabla u, \nabla v) - (p, \dive v)
        &=
        \langle f, v \rangle
        \quad\forall v,
        \\
        (\dive u, q) &= 0 \hspace{3em}\forall q.
    \end{cases}
    \end{displaymath}
    By inspection we see that \(A=-\Delta\), \(B=\dive\).

    \item \textbf{Darcy (or mixed Poisson)}: We are dealing 
    with the problem
    \begin{displaymath}
        -\Delta u = f\quad \text{ in } \Omega
    \end{displaymath}
    Introducing the variable \(\sigma = -\nabla u\), the problem
    now reads:
    \begin{displaymath}
    \left\{
    \begin{aligned}
        \dive \sigma &= f && \quad \text{ in } \Omega\\ 
        \sigma + \nabla u &= 0 && \quad \text{ on } \partial\Omega
    \end{aligned}
    \right.
    \end{displaymath}
    Testing the equation on \(\tau\) and :
    \begin{displaymath}
    \left\{
    \begin{aligned}
        (\sigma, \tau) - (u, \dive \tau) &= 0
        &&\quad\forall \tau, \\ 
        (\dive \sigma, v) &= \langle f,v \rangle
        &&\quad \forall v,
    \end{aligned}
    \right.
    \end{displaymath}
    where $A$ is the identity operatir and $B=\dive$ as before.

    \item \textbf{Primal-Mixed Poisson (Dirichlet with multipliers)}
    Consider the problem
    \begin{displaymath}
    \left\{
    \begin{aligned}
        -\Delta u &= f &&\quad\text{ in } \Omega \\ 
        u &= g &&\quad\text{ on } \partial\Omega
    \end{aligned}
    \right.
    \end{displaymath}
    Integration by parts yields:
    \begin{displaymath}
        (\nabla u, \nabla v) 
        - \langle \gamma_N u, \gamma_D v \rangle
        = \langle f,v \rangle.
    \end{displaymath}
    Define \(\xi = -\gamma_N u\) and impose Dirichlet boundary
    conditions weakly. That is,
    \begin{displaymath}
        \forall \lambda \in H^{-1/2}(\partial\Omega)
        \colon\qquad
        \langle \lambda, \gamma_D u \rangle
        =
        \langle \lambda, g \rangle
    \end{displaymath}
    Writing everything together shows a saddle point problem:
    \begin{displaymath}
    \left\{
    \begin{aligned}
        (\nabla u, \nabla v) + \langle \xi, \gamma_D v \rangle
        &= \langle f,v \rangle
        &&\quad \forall v,\\ 
        \langle \lambda, \gamma_D u \rangle &= \langle \lambda, g \rangle
        &&\quad  \forall \lambda.
    \end{aligned}
    \right.
    \end{displaymath}
\end{itemize}

Now that we know some examples of saddle points problems, a natural
question is to ask for conditions for the existence and uniqueness
of solutions. Fortunately, this has already been done and it's known
as the \emph{Ladyzhenskaya-Babu\v{s}hka-Brezzi Theory}, typically denoted as LBB theory.

\begin{theorem}
    Consider the problem
    \begin{equation*}\label{mixed}
        \begin{bmatrix}
            A & B^{T}\\ 
            B & 0
        \end{bmatrix}
        \begin{bmatrix}
            u \\ p
        \end{bmatrix}
        =
        \begin{bmatrix}
            f \\ g
        \end{bmatrix}
    \end{equation*}
    Let \(V = \ker B\) and \(\Pi\colon H\to V\) the orthogonal
    projector. Suppose that
    \begin{itemize}
        \item \(\Pi A\colon V\to V\) is a bijection;
        \item the bilinear form \(b\) (associated to \(B\))
        satisfies the inf-sup condition with constant \(\beta\).
    \end{itemize}
    Then, for each \((f,g)\) in \(H'\times Q'\) there exists
    a unique pair \((u,p)\in H\times Q\) such that~\eqref{mixed}
    holds. Moreover, there is a positive constant
    \(C=C(\|A\|, \| (\Pi A)^{-1}\|, \beta)\) such that
    \begin{equation*}
        \| (u,p) \|
        \le
        C \left( \|f\| + \|g\| \right) 
    \end{equation*}
\end{theorem}
\begin{proof}
    Exercise :)
\end{proof}

%%%%%%%%%%%%%%%%%%%%%%%%%%%%%%%%%%%%%%%%%%%%%%%%%%%%

\paragraph{Common example: Darcy} We will present a the worked example of Darcy's problem, to be analyzed using the LBB theory. Its weak form is given as follows: Consider $\vec H=H(\dive; \Omega)\cap \{\vec u\cdot \vec n=0\}$ and $Q=L^2(\Omega)$, then find $(u,p)$ in $H\times Q$ such that
    \[ \begin{aligned}
    (\ten K^{-1}\vec u, \vec v) - (\dive \vec v, p) &= \langle f,\vec v\rangle && \forall \vec v\in \vec H \\
    (\dive \vec u, q)  &= \langle g, q\rangle && \forall q \in Q,
    \end{aligned} \]
where $\ten K$ is symmetric and positive definite ($ k_1|\vec x|^2 \leq \vec x\cdot \ten K^{-1}\vec x \leq k_2 |\vec x|^2$), $f$ is in $\vec H'$ and $g$ is in $Q'$. We omit details regarding continuity as it is a simply application of the Cauchy-Schwartz inequality. The bilinear forms to be studied here are $a(\vec u, \vec v) = (\ten K^{-1}\vec u, \vec v)$ and $b(\vec v, q) = (\dive \vec v, q)$.
    \begin{itemize}
        \item First, we need to show that $\Pi A|_V$ is invertible. To better understand this operator, we recall that $\Pi$ is the orthogonal projector of $\vec H$ into $\vec V = \ker B$, so let's look into all the pieces to make sense out of it. First, $B:H\to Q$ is the operator given by 
        \begin{equation*}
(B\vec v, q) = (\dive \vec v, q),
\end{equation*}
        and thus, as $L^2$ can be identified with its dual, we have that simply $B = \dive$ and thus $\vec V = \{\vec v \in \vec H: \dive \vec v = 0\}$, i.e. the space of solenoidal functions in $H(\dive;\Omega)$ with null normal component on the boundary. Second, the projection $\Pi$ is surjective, and thus 
            \[ \langle \Pi A|_V \vec u, \vec v\rangle \]
        which is defined for all $\vec u, \vec v$ in $\vec H$, can be written analogously as 
            \[ \langle \Pi A \vec u, \vec v\rangle \]
        for $\vec u$ in $\vec V$ and $\vec v$ in $\vec H$, where $A\vec u$ belongs to $\vec H$. Finally, $\Pi:\vec H\to\vec V$ is an orthogonal projector, meaning that if we denote $\vec H = \vec V \oplus \vec V^\perp$ and $\vec v = \vec v_0 + \vec v^\perp$, we obtain
            \[  \langle A \vec u, \vec v_0\rangle \]
        with $\vec u$ in $\vec V$ and $\vec v_0$ in $\vec V$. In other words, the operator $\Pi A|_V$ is simply the restricted bilinear form $a: \vec V\times \vec V \to \R$, given by
        \[ a(\vec u, \vec v) = (\ten K^{-1}\vec u, \vec v) \qquad\forall \vec u, \vec v \in \vec V. \]
        In such a space, the form $a$ is elliptic: 
        \[ a(\vec u, \vec u) = (\ten K^{-1}\vec u, \vec u) \geq k_1 \| \vec u \|_0^2 = k_1 \| \vec u\|_{\dive}^2 \qquad\forall \vec u \in \vec V,\]
        where the term $\|\dive \vec u\|$ can be trivially added as $\dive \vec u = 0$ for $\vec u$ in $\vec V$. This shows that $\Pi A|_V$ is invertible using the Lax-Milgram lemma.
        \item We now show the inf-sup property. For this, we will use the auxiliary problem technique. Consider the following problem
        \[ \begin{aligned}
            - \Delta z &= q && \Omega \\
            \grad z\cdot \vec n &= 0 &&\partial\Omega,
        \end{aligned} \]
        which we have already shown to be invertible in the subspace of $H^1$ that is orthogonal to the constants. Thus, it holds that the function $\tilde{\vec u} = -\grad z$ is such that it belongs to $\vec H$ and satisfies that $\|\tilde{\vec u}\|_0 \leq C \| q\|_0$, which comes from the \emph{a-priori} (or stability) estimate. Also, as $\dive \tilde{\vec u} = q$, we have $\|\tilde{\vec u}\|_{\dive} \leq (C+1)\|q\|$. Going back to the original problem, we want to show that there exists $\beta>0$ such that
        \[ \sup_{\vec v\in \vec H}\frac{b(\vec v, q)}{\|\vec v\|} \geq \beta \|q\|_0 \qquad \forall q \in Q.\]
        Using our previously constructed solution, we obtain
        \[ \sup_{\vec v\in \vec H}\frac{b(\vec v, q)}{\|\vec v\|} \geq \frac{(\dive \tilde{\vec u}, q)}{\|\tilde{\vec u}\|_{\dive}}\geq \frac{\|q\|_0^2}{(C+1)\|q\|} = \tilde\beta \|q\|, \]
        where $\tilde \beta = 1/(C+1)$. This concludes the proof. We note that it was actually sufficient to show that for each $q$ there existed an element $\tilde{\vec u}$ in the desired space, as it proved the survectivity of the operator $B$, but showing the complete inf-sup estimate was more instructive. 
    \end{itemize}
    Given that we have shown the required properties for $a$ and $b$, then there exists a unique and stable solution $(u,p)$ of Darcy's problem. 

%%%%%%%%%%%%%%%%%%%%%%%%%%%%%%%%%%%%%%%%%%%%%%%%%%%%
\section{Discretization of saddle point problems}
%%%%%%%%%%%%%%%%%%%%%%%%%%%%%%%%%%%%%%%%%%%%%%%%%%%%
We consider finite dimensional and conforming spaces $\{H_h\}_h \subset H$ and $\{Q_h\}_h \subset Q$. Then, given $\vec F$ in $H'$ and $G$ in $Q'$, the discrete problem reads: Find $(u_h, p_h)$ in $H_h \times Q_h$ such that 
    \[ \begin{aligned}
        a(u_h, v_h) + b(v_h, p_h) &= \langle F, v_h\rangle &&\forall v_h \in H_h \\
        b(u_h, q_h)  &= \langle G, q_h\rangle &&\forall q_h \in Q_h. 
    \end{aligned} \]
The previous theory can be used in this context unchanged, with only some mild changes in the definition of the operators, which we do in the following before announcing the operators. Consider the induced operators $A_h:H_h\to H_h$ and $B_h: H_h\to Q_h$, defined using convenient Riesz operators as done previously, and define the kernel space $V_h = \ker B_h = \{v_h \in H_h: b(v_h, q_h) = 0 \quad\forall q_h \in Q_h\}$. 
\begin{theorem} Consider the orthogonal projection $\Pi_h: H_h \to Q_h$. Then, if 
    \begin{itemize}
        \item The operator $\Pi_h A_h|_{V_h}: V_h\to V_h$ is injective (or surjective), and
        \item the bilinear form $b:H_h\times Q_h\to \R$ satisfies and inf-sup condition, then
    \end{itemize}
for each pair of functions $(F, G)$ there is a unique solution $(u_h, p_h)$ in $H_h\times Q_h$ such that 
    \[ \| (u_h, p_h) \|_{H\times Q} \leq C_h\left( \|F|_{H_h}\|_{H_h'} + \| G|_{Q_h} \|_{Q_h'} \right), \]
where $C_h = C_h(\| A_h\|, \| (\Pi_h A)^{-1} \|, \beta_h)$. 
\end{theorem}
From this result, it is easy to see that the Galerkin projection $G_h: H\times Q \to H_h\times Q_h$ is well-posed: 
    \[\begin{aligned}
        a(\Pi_H \circ G_h(u,p), v_h) + b(v_h, \Pi_Q \circ G_h(u,p)) &= a(u, v_h) + b(p, v_h) && \forall v_h \in H_h\\
        b(\Pi_H \circ G_h(u,p), q_h) &= b(u, q_h) && \forall q_h \in Q_h,
    \end{aligned}\]
where $\Pi_H$ and $\Pi_Q$ are simply component projections, i.e. $\Pi_H(u,p) = u$ and $\Pi_Q(u,p) = p$. One can further prove a Céa estimate: 
    \[ \| u - u_h\| \leq C_1 \|\inf_{\zeta_h \in H_h}\| u -\zeta_h \|_H + C_2\inf_{w_h\in Q_h}\| p -w_h\|\]
    \[ \|p - p_h\| \leq C_3 \|\inf_{\zeta_h \in H_h}\| u -\zeta_h \|_H + C_4\inf_{w_h\in Q_h}\| p -w_h\|.\]
Note that the discrete inf-sup condition \emph{does not} follow from the continuous one, so it is typically an additional difficulty during the analysis. Still, there is a classical lemma that allows to infer the discrete inf-sup in some conditions. 

\begin{lemma}[Fortin's Lemma]
Consider $b:H\times Q\to \R$ that satisfies an inf-sup condition with constant $\beta >0$. If there exists a famliy of discrete projectors $\Pi_h: H\to H_h$ such that 
    \[ \|\Pi_h \|\leq \tilde C \quad \forall h \qquad \text{ and } \qquad b(\Pi_h u, q_h) = b(u, q_h) \quad\forall u\in H, q_h \in Q_h,\]
then the discrete inf-sup of $b:H_h\to Q_h$ holds with $\tilde \beta = \beta / \tilde C$.
\end{lemma}


